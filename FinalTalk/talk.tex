\documentclass{beamer}
\usetheme{Madrid} % My favorite!
%\usetheme{Boadilla}
% Pretty neat, soft color.
%\usetheme{default}
%\usetheme{Warsaw}
%\usetheme{Bergen}
% This template has nagivation on the left
%\usetheme{Frankfurt} % Similar to the default
% with an extra region at the top.
%\usecolortheme{seahorse} % Simple and clean template
%\usetheme{Darmstadt} % not so good
% Uncomment the following line if you want
% % page numbers and using Warsaw theme%
% \setbeamertemplate{footline}[page number]
%\setbeamercovered{transparent}
\setbeamercovered{invisible} % To remove the navigation symbols from
% the bottom of slides
% \setbeamertemplate{navigation symbols}{}
\usepackage{graphicx}
\usepackage{calc}

\newsavebox\CBox
\newcommand\hcancel[2][0.5pt]{%
  \ifmmode\sbox\CBox{$#2$}\else\sbox\CBox{#2}\fi%
  \makebox[0.5pt][l]{\usebox\CBox}%  
  \rule[0.8\ht\CBox-#1/2]{\wd\CBox}{#1}}
  
\makeatletter
\newenvironment<>{proofs}[1][\proofname]{%
    \par
    \def\insertproofname{#1\@addpunct{.}}%
    \usebeamertemplate{proof begin}#2}
  {\usebeamertemplate{proof end}}
\makeatother


\def\qd{\,{\mathchar'26\mkern-12mu d}}
  
  

%\usepackage{bm} % For typesetting bold math (not \mathbold)
%\logo{\includegraphics[height=0.6cm]{yourlogo.eps}} %
\title[Harmonic Analysis]{Quantised Calculus in One Dimension}
\author[McDonald, E.]{McDonald, E. \\Supervisor: Sukochev, F.}
\institute[UNSW] { UNSW Australia }
%\titlegraphic{\includegraphics[width=60mm]{}
\date{\today}
\newcommand{\Rl}{\mathbb{R}}
\newcommand{\Cplx}{\mathbb{C}}
\newcommand{\Circ}{\mathbb{T}}
\newcommand{\Itgr}{\mathbb{Z}}
\newcommand{\Ntrl}{\mathbb{N}}
\newcommand{\ha}{\boldsymbol{m}}
\newcommand{\A}{\mathcal{A}}
\newcommand{\Hilb}{\mathcal{H}}
\newcommand{\D}{\mathcal{D}}
%\newcommand{\qd}{\hcancel[0.9pt]{d}}
\newcommand{\sgn}{\operatorname{sgn}}
\newcommand{\rk}{\operatorname{rank}}
\newcommand{\BMO}{\operatorname{BMO}}
\newcommand{\VMO}{\operatorname{VMO}}
\newcommand{\Proj}{\mathbf{P}}
\newcommand{\Ban}{\mathbf{Ban}}
% \today will show current date.
% Alternatively, you can specify a date.
\begin{document}
\begin{frame}
\titlepage
\end{frame} %


\begin{frame}
\frametitle{Introduction}
The purpose of this talk is to introduce the \emph{quantised calculus},
which is a tool coming from non-commutative geometry
which gives rigorous justification to computations involving ``infinitesimals".


Many thanks go to my supervisor, who has put up with me for a very long time.
\end{frame}

\begin{frame} 
\frametitle{Infinitesimals} 
\begin{itemize}
    \item{} Early calculus (e.g. Leibniz, Newton) made use of infinitesimal quantities:
\end{itemize}
\begin{definition}
    A quantity $x$ is called \emph{infinitesimal} if for any integer $n > 0$,
    \begin{equation*}
        |x| < \frac{1}{n}.
    \end{equation*}
\end{definition}
\begin{definition}
    A quantity $x$ is \emph{infinite} if for any integer $n > 0$, 
    \begin{equation*}
        |x| > n.
    \end{equation*}
\end{definition}
\end{frame} 

\begin{frame} 
\frametitle{Infinitesimals}
This definition was good enough for $18$th century calculus, and many
definitions were formulated in terms of infinitesimals.

For example:
\begin{definition}
    A function $f$ is continuous if $df(x) := f(x+dx)-f(x)$ is infinitesimal
    for any infinitesimal quantity $dx$.
\end{definition}
\begin{definition}
    A function $f$ is differentiable if the quantity
    \begin{equation*}
        f'(x) := \frac{f(x+dx)-f(x)}{dx}
    \end{equation*}
    is not infinite when $dx$ is infinitesimal.
\end{definition}
%\includegraphics[width=90mm]{Infinitesimal_Calculus_6.png}
\end{frame}

\begin{frame}
\frametitle{Properties of infinitesimals}
\begin{lemma}
    If $x$ is infinitesmal, then $x^{-1}$ is infinite.
\end{lemma} 
\begin{lemma}
    For any $f$,
    \begin{equation*}
        df = \frac{df}{dx}dx.
    \end{equation*}
\end{lemma}
\end{frame}

\begin{frame}
\frametitle{Sizes of infinitesimals}
Not all infinitesimals are equal.
Intuitively we expect the following properties:
\begin{itemize}
\item{} If $x > 0$ is infinitesimal, then $x^2 < x$.
\item{} If a function $f$ is smoother than a function $g$, then $df < dg$... in some
sense.
\end{itemize}
\end{frame}

\begin{frame}
\frametitle{Problem}
There is a problem! These definitions make no sense.
\begin{block}
{Problems:}
\begin{itemize}
    \item{} If $x$ is infinitesimal, then $x = 0$.

    \item{} If $x$ is infinite, then $|x| > 2|x|$, which is absurd.
\end{itemize}
\end{block}

So we cannot use the definitions given by $18$th century mathematicians.
\end{frame}


\begin{frame}
    Infinitesimals were rightly banished from mathematics, and the definitions
    of continuity and differentiability were replaced with their modern 
    definitions in terms of limits.
\end{frame}

\begin{frame}
\frametitle{But what \emph{is} an infinitesimal?}
$18$th century mathematicians were still able to use infinitesimals
even though they make no sense. 
\begin{block}
    {Question:}
    Why does calculus using infinitesmals ``work"?
\end{block}
\begin{block}
    {Question:}
    Is there a way to make sense of infinitesimals?
\end{block}
\end{frame}



\begin{frame}
\frametitle{Introducing Non-commutative Geometry}
\begin{block}{What is noncommutative geometry?}
The study of noncommutative algebras which resemble algebras of functions
on geometric spaces, using the methods and the language of geometry.
\end{block}
\end{frame}

\begin{frame}
\frametitle{Non-commutative geometry}
Non-commutative geometry in analysis is usually the study of algebras of operators on
Hilbert space. Algebras of operators are considered the generalisation of
algebras of functions.

\begin{example}
    The algebra $L^\infty(\Rl)$ is an algebra of bounded operators
    on $L^2(\Rl)$. The full algebra $\mathcal{B}(L^2(\Rl))$ is the
    ``noncommutative extension" of the study of functions on $\Rl$.
\end{example}
\end{frame}

\begin{frame}
\frametitle{Question:}
We would like to think of all objects of interest as being operators
on a Hilbert space.

Does this give us a way to define infinitesimals?
\end{frame}

\begin{frame}
\frametitle{Infinitesimal Operators}
Let $\Hilb$ be a (complex, separable) Hilbert space.
An infinitesimal operator $T \in \mathcal{B}(\Hilb)$ should be an operator
$T$ such that for any $\varepsilon > 0$, we have
\begin{equation*}
    \|T\| < \varepsilon.
\end{equation*}

Again, this definition is useless as it implies that $T = 0$.

\end{frame}

\begin{frame}
\frametitle{Compact Operators as Infinitesimals}
We shall say that an operator $T \in \mathcal{B}(\Hilb)$
is \emph{infinitesimal} if for any $\varepsilon > 0$
there exists a finite dimensional subspace $E$
such that
\begin{equation*}
    \|T|_{E^\perp}\| < \varepsilon.
\end{equation*}
This is equivalent to saying that $T$ is compact.
\end{frame}



\begin{frame}
\frametitle{Sizes of Infinitesimals}
\begin{definition}
    Given $T \in \mathcal{B}(\Hilb)$, define the $k$th
    singular value of $T$ to be
    \begin{equation*}
        \mu_k(T) := \inf\{\|T-A\|\;:\;\rk(A) \leq k\}.
    \end{equation*}
\end{definition}
For a compact operator, $\{\mu_k(T)\}_{k=0}^\infty$ is a
vanishing sequence of positive numbers. We
shall describe the \emph{size} of an infinitesimal
$T$ as the \emph{rate of decay} of $\{\mu_k(T)\}_{k=0}^\infty$.

\end{frame}


\begin{frame}
\frametitle{Sizes of Infinitesimals}
We can quantify the sizes an infinitesimal $T$ by placing conditions
on the rate of decay of $\{\mu_k(T)\}_{k=0}^\infty$.
\begin{itemize}
    \item{} The smallest infinitesimals have $\{\mu_k(T)\}_{k=0}^\infty$
    of finite support. Then $T$ is of finite rank.
    \item{} We say that $T \in \mathcal{L}^p$ if $\{\mu_k(T)\}_{k=0}^\infty \in \ell^p$.
    \item{} We say that $T \in \mathcal{L}^{p,\infty}$ if $\mu_k(T) = \mathcal{O}(k^{-1/p})$.
    \item{} We say that $T \in \mathcal{L}^{p,q}$ if $\{k^{1/p-1/q}\mu_k(T)\}_{k=0}^\infty \in \ell^q$.
    \item{} We say that $T \in \mathcal{M}_{1,\infty}$ if $\{\frac{1}{\log(k+1)}\sum_{n=0}^k \mu_k(T)\}_{k=0}^\infty \in \ell^\infty$.
\end{itemize}
These last four conditions in fact correspond to ideals of $\mathcal{B}(\Hilb)$.
\end{frame}



\begin{frame}
\frametitle{Sizes of Infinitesimals}
\begin{theorem}
    Let $T$ be a compact operator in $\mathcal{B}(\Hilb)$. Then
    for every $k \geq 0$,
    \begin{equation*}
        \mu_k(T^2) \leq \|T\|\mu_k(T).
    \end{equation*}
\end{theorem}
Hence, if $T$ is an infinitesimal, then $T^2$ is a smaller infinitesimal.
\end{frame}


\begin{frame}
\frametitle{Quantised Differentials}
In noncommutative geometry, we have a new object
that is not present in classical analysis called a quantised
derivative or quantised differential.
\begin{definition}
    Let $f \in L^\infty(\Rl)$. $M_f$ is the operator
    on $L^2(\Rl)$ of pointwise multiplication: $M_fg(x) = f(x)g(x)$
    for almost all $x \in \Rl$. $F$ denotes the Hilbert transform:
    \begin{equation*}
        Fg(x) := \frac{1}{\sqrt{2\pi}}\int_\Rl \sgn(\xi)e^{ix\xi}\widehat{g}(\xi)\;d\xi.
    \end{equation*}
    (Defined at least initially for $g$ a smooth function of compact support, then extended by continuity
    to $g \in L^2(\Rl)$).
    Then we define:
    \begin{equation*}
        \qd f := [F,M_f].
    \end{equation*}
\end{definition}
\end{frame} 

\begin{frame}
\frametitle{What is a quantised differential?}
$\qd f$ is supposed to represent the ``infinitesimal variation", like $df$
in classical analysis.
\begin{block}{Warning:}
    $\qd f$ is not a one-form, or a derivative. It is an infinitesimal 
    deviation. It should be thought of as $f(x+h)-f(x)$
    for infinitesimal epsilon.
\end{block}
It is very hard to motivate the definition of $\qd f$. Instead, we will show
that is satisfies a number of ``headline properties" of a classical differential.
\end{frame}



\begin{frame}
\frametitle{A dictionary}

\begin{tabular}{l | r}
Classical Analysis & Non-commutative Analysis\\
\hline
Function & Operator\\
Range & Spectrum\\
Infinitesimal & Compact Operator\\
Differential & Quantised differential\\
\end{tabular}
\end{frame}

\begin{frame}
\frametitle{Headline properties of infinitesimals}
In classical $17$-$19$th century analysis, infinitesimals
were supposed to have a number of properties:
\begin{enumerate}
    \item{} If $f:\Rl\rightarrow \Rl$ is a function, there
    is a function $df$ representing
    infinitesimal variation in $f$. $f$ is continuous
    if and only if $df$ is infinitesimal.
    \item{} If $f$ is smoother than $g$, then $df$ is smaller
    than $dg$.
    \item{} If $x$ is a positive infinitesimal, then $x^2$
    is smaller than $x$.
    \item{} If $f$ is a differentiable function, then $df = f' dx$,
    provided that sufficiently small infinitesimals
    are ignored. 
\end{enumerate}
We shall see that the quantised differential satisfies all
of these conditions, if they are interpreted correctly.
\end{frame}

\begin{frame}
\frametitle{Smoothness and rate of decay}
In classical analysis, we had the statement
that if $f$ is smoother than $g$, then $df$ is smaller than $dg$.
\begin{block}
{The main question of this talk:}
In what sense is it true that if $f$ is smoother than $g$,
then $\qd f$ is smaller than $\qd g$?
\end{block}
We shall restrict attention to functions on the circle $\Circ$
and the line $\Rl$.
\end{frame}



\begin{frame}
\frametitle{Revision of Classical Fourier Analysis and Notation}
We define $\Circ = \{\zeta \in \Cplx\;:\;|\zeta| = 1\}$.
Let $z:\Circ\to\Circ$ be the identity function. 
Denote the normalised Haar (or arc length) measure on $\Circ$ by $\ha$. 

For $f \in L^1(\Circ,\ha)$, define for $n \in \Itgr$,
\begin{equation*}
    \widehat{f}(n) := \int_\Circ z^{-n} f\;d\ha.
\end{equation*}

Recall that any $f \in L^2(\Circ,\ha)$ can be written as
\begin{equation*}
    f = \sum_{n \in \Itgr} \widehat{f}(n)z^n.
\end{equation*}
The sum converges in the $L^2$ sense. This effects
an isometric isomorphism between $L^2(\Circ)$ and $\ell^2(\Itgr)$.
\end{frame}

\begin{frame}
\frametitle{Revision of Classical Fourier Analysis and Notation}
The closed linear span of $\{z^n\}_{n=0}^\infty$ in $L^2$ is denoted
$H^2(\Circ)$, and the orthogonal complement is denoted $H^2_-(\Circ)$.

We define the space of polynomials $\mathcal{P}(\Circ)$
to be the finite linear span of $\{z^n\}_{n\in \Itgr}$. 
$P_A(\Circ) = \operatorname{span}\{z^n\}_{n\geq 0}$. 
\end{frame}



\begin{frame}
\frametitle{Differentials on $\Circ$}
Defining quantised differentials for functions on $\Circ$ is exactly
like for functions on $\Rl$.
\begin{definition}
    The Hilbert transform, for $g \in L^2(\Circ)$, is defined to be
    \begin{equation*}
        Fg := \sum_{n \in \Itgr} \sgn(n)\widehat{g}(n)z^n.
    \end{equation*}
    We define the quantised differential:
    \begin{equation*}
        \qd f := [F,M_f].
    \end{equation*}
\end{definition}
\end{frame}




\begin{frame}
\frametitle{Differentials on $\Circ$}
Hence for $f \in L^2(\Circ)$, 
\begin{equation*}
    Ff = \varphi * f
\end{equation*} 
where
\begin{equation*}
    \varphi = \sum_{n\in\Itgr} \sgn(n)z^n = \frac{1}{1-z}-\frac{z^{-1}}{1-z^{-1}} = \frac{2}{1-z}.
\end{equation*}

   Thus, 
\begin{align*}
    (\qd f)g &= ([F,f]g)(t) \\
    &= 2\lim_{\varepsilon \to 0} \int_{|\tau-t| > \varepsilon} \frac{f(t)-f(\tau)}{t-\tau}g(\tau)\;d\ha(\tau).
\end{align*} 
\end{frame}


\begin{frame}
\frametitle{Finite rank differentials}
Let $f:\Circ\rightarrow \Cplx$. The strictest
condition we can put on the smoothness
of $f$ is that $f$ is a rational function.
The strictest condition we can put on the size of $\qd f$
is that $\qd f$ is finite rank.
These two conditions are equivalent.
\begin{theorem}[Kronecker]
    If $f:\Circ\rightarrow \Cplx$, then $\qd f$ is finite rank
    if and only if $f$ is a rational function.
\end{theorem}   
\end{frame}

\begin{frame}
\frametitle{Bounded differentials}
Let $f:\Circ\rightarrow \Cplx$. The weakest condition
that we can place on $\qd f$ is that $\qd f$ is bounded.
\begin{definition}
    Let $f:\Circ\rightarrow \Cplx$ be measureable. We say that $f$ is of \emph{bounded mean
    oscillation} if for an arc $I \subseteq \Circ$, define
    \begin{equation*}
        f_I = \frac{1}{\ha(I)}\int_I f\;d\ha
    \end{equation*}
    and
    \begin{equation*}
        \sup_{I} \frac{1}{\ha(I)}\int_I |f-f_I|\;d\ha < \infty
    \end{equation*}
    where the supremum runs over all arcs $I$.
    The set of functions with bounded mean oscillation is denoted $\BMO(\Circ)$.
\end{definition}
\end{frame} 

\begin{frame}
\frametitle{Bounded differentials}
\begin{theorem}[Nehari]
    Let $f:\Circ\rightarrow \Cplx$. Then $\qd f$ is bounded
    if and only if $f \in \BMO(\Circ)$.
\end{theorem}
\end{frame}

\begin{frame}
\frametitle{Compact differentials}
We define the space $\VMO(\Circ)$:
\begin{definition}
    We say that $f \in \VMO(\Circ)$ if $f \in \BMO(\Circ)$
    and
    \begin{equation*}
        \lim_{\ha(I)\rightarrow 0} \frac{1}{\ha(I)}\int_I |f-f_I|\;d\ha = 0.
    \end{equation*}
\end{definition}
\begin{theorem}
    If $f:\Circ\rightarrow \Cplx$, then $\qd f$ is compact if and only if $f \in \VMO(\Circ)$.
\end{theorem}
\end{frame}

\begin{frame}
\frametitle{Can we do better?}
We seek a more precise characterisation of the relationship between
the smoothness
of $f$ and the size of $\qd f$. To this end, we define the \emph{Besov classes} $B^s_{pq}$.
\end{frame}

\begin{frame}
\frametitle{Definition of $B_{pq}^s$} 
\begin{definition}
    Let $\tau \in \Circ$, and $f \in L^1(\Circ)$. Define
    \begin{equation*}
        \Delta_\tau f(x) := f(x\tau)-f(x).
    \end{equation*}
    Let $s > 0$, and $p,q \in [1,\infty]$. Let
    $n > s$ be an integer.
    The Besov space $B_{pq}^s$ is defined to be:
    \begin{equation*}
        B^s_{pq}(\Circ) := \left\{f \in L^p(\Circ)\;:\; \int_\Circ \frac{\|\Delta^n_\tau f\|_p^q}{|1-\tau|^{1+sq}}d\ha(\tau) < \infty\right\}, q < \infty
    \end{equation*}
    \begin{equation*}
        B^s_{p\infty}(\Circ) := \left\{f\in L^p(\Circ)\;:\;\sup_{\tau \neq 1}\frac{\|\Delta^n_\tau f\|_p}{|1-\tau|^s} < \infty\right\}
    \end{equation*}
\end{definition}
\end{frame}

\begin{frame}
\frametitle{Quantised differentials of Schatten-Von Neumann class}
\begin{theorem}[Peller]
    Let $f$ be a measurable function on $\Circ$
    and $p > 0$. Then $\qd f \in \mathcal{L}^p$ if and only
    if $f \in B_{pp}^{1/p}(\Circ)$.
\end{theorem}
\end{frame}



\begin{frame}
\frametitle{Can we do better?}
What about finding conditions on $\varphi$ such that $\qd \varphi \in \mathcal{L}^{p,q}$?

This problem can be solved with an interpolation functor: 
\begin{equation*}
    F(\mathcal{L}^1,\mathcal{B}(L^2(\Circ))) = \mathcal{L}^{p,q}.
\end{equation*}
Hence, if $\varphi \in F(B_{11}^1,\BMO(\Circ))$, then
\begin{equation*}
    \qd \varphi \in \mathcal{L}^{p,q}.
\end{equation*}
There indeed exists such a functor: $F = K(-,-)_{(1-p)^{-1},q}$. This is
highly technical and explained in detail in the thesis.
\end{frame}


\begin{frame}
\frametitle{Summary}
In summary we have:\\
\begin{tabular}{l | r}
$f$ & $\qd f$\\
\hline
Rational & Finite rank\\
$B_{pp}^{1/p}$ & $\mathcal{L}^p$\\
$K(B_{11}^1,VMO)_{(1-p)^{-1},q}$ & $\mathcal{L}^{p,q}$\\
$\VMO$ & Compact\\
$\BMO$ & Bounded
\end{tabular}
\\
You should now be convinced that the statement ``If $f$ is smoother than $g$,
then $\qd f$ is smaller than $\qd g$" has at least some rigorous justification
for functions on $\Circ$.
\end{frame}


\begin{frame}
\frametitle{Quantised differentials on $\Rl$}
So far we have talked about quantised differentials exclusively on $\Circ$. However,
many of the same results hold for $\Rl$, since we have the \emph{Cayley transform}.

Let $\zeta \in \Circ\setminus \{1\}$. Let 
\begin{equation*}
    \omega(\zeta) = i\frac{1+\zeta}{1-\zeta}.
\end{equation*}

Let $g \in L^0(\Rl)$. Define for $\zeta \in \Circ$, 
\begin{equation*}
    \mathcal{C}(g)(\zeta) = \frac{\sqrt{\pi}}{2i}\frac{(g\circ \omega)(\zeta)}{1-\zeta}
\end{equation*}
\end{frame}

\begin{frame}
\frametitle{Quantised differentials on $\Rl$}
\begin{theorem}
    $\mathcal{C}:L^2(\Rl)\to L^2(\Circ)$, and $\mathcal{C}$
    is unitary.
    
    Moreover, for $f \in L^2(\Rl)$,
    \begin{equation*}
        \mathcal{C}\qd f \mathcal{C}^{-1} = \qd (f\circ \omega).
    \end{equation*}
\end{theorem}
Hence results on $\Circ$ can be transferred to results on $\Rl$.
\end{frame}



\begin{frame}
\frametitle{Proving these statements}
To prove all the preceding results, we need to introduce the theory of Hankel operators.
\end{frame}



\begin{frame}
\frametitle{Hankel operators} 
\begin{definition}
    Let $\varphi \in L^1(\Circ)$, and $M_\varphi$
    is the densely defined pointwise multiplication operator
    on $L^2(\Circ)$. Let $\Proj_-$ be the projection,
    \begin{equation*}
        \Proj_-\left(\sum_{n \in \Itgr} a_nz^n\right) = \sum_{n < 0} a_nz^n.
    \end{equation*}
    Then $H_\varphi := \Proj_-M_\varphi|_{H^2(\Circ)}$
\end{definition}   
\end{frame}

\begin{frame}
\frametitle{Hankel operators and quantised differentials} 
    The link between Hankel operators and quantised differentials is provided
    by the following result:
    \begin{theorem}
        Let $\varphi \in L^1(\Circ)$. Let $\varphi_- = \Proj_-\varphi$ and $\varphi_+ = (1-\Proj_-)\varphi$. Then,
        \begin{equation*}
            \qd \varphi = 2((H_{\varphi_+})^*-H_{\varphi_-})
        \end{equation*}
    \end{theorem}
    
Since the two Hankel operators act on orthogonal complements, we
see that $\qd \varphi$ is bounded if and only if $H_{\varphi_+}$
and $H_{\varphi_-}$ are bounded.
\end{frame}



\begin{frame}
\frametitle{Reformulation in terms of $H_\varphi$}
Recall:
\begin{equation*}
    H_\varphi = \Proj_- M_\varphi|_{H^2(\Circ)}.
\end{equation*}
For $\varphi \in L^1(\Circ)$. This has matrix
representation $\{\widehat{\varphi}(-j-k)\}_{j\geq 1,k\geq 0}$. 
\end{frame}


\begin{frame}
\frametitle{Hankel Operators}
A \emph{Hankel matrix} is an infinite matrix $\{a_{j+k}\}_{j,k \geq 0}$,
\begin{equation*}
    \begin{pmatrix}
        a_0 & a_1 & a_2 & a_3 & \cdots\\
        a_1 & a_2 & a_3 & a_4 & \cdots\\
        a_2 & a_3 & a_4 & a_5 & \cdots\\
        a_3 & a_4 & a_5 & a_6 & \cdots\\
        \vdots & \vdots & \vdots & \vdots & \ddots
    \end{pmatrix}
\end{equation*}
    All of our results about quantised differentials come from
    the (very old) theory of Hankel matrices.
\end{frame}

\begin{frame}
\frametitle{Quantised differentials and classical derivatives}
Recall that $17$th century differentials were supposed to have the property
that 
\begin{equation*}
    df = f'dx
\end{equation*}
provided that sufficiently small infinitesimals are ignored. 


We can now interpret this in a rigorous sense. First, we replace
infinitesimals with quantised differentials. To ``ignore sufficiently small infinitesimals" means
that we work modulo some ideal of compact operators.
\end{frame}


\begin{frame}
\frametitle{Quantised differentials and classical derivatives}
Let $\mathcal{L}_0^{p,\infty}$ be the closure in $\mathcal{L}^{p,\infty}$
of the ideal of finite rank operators. 
\begin{theorem}[Connes]
    Let $f \in C(\Circ)$ be such that $\qd f \in \mathcal{L}^{p,\infty}$ and $\varphi \in C^\infty(\sigma(f))$, and $p \in [1,\infty)$. Then
    \begin{equation*}
        \qd \varphi(f) \equiv \varphi'(f)\qd f\;\operatorname{mod}\;\mathcal{L}^{p,\infty}_0.
    \end{equation*}
\end{theorem}
\end{frame}

\begin{frame}
\frametitle{Further results}
What more can be done?
\begin{block}{Better ideal membership results}
    Can we find conditions on $\varphi$ (a function either on $\Circ$
    or $\Rl$) such that $\qd \varphi \in \mathcal{M}_{1,\infty}$?
\end{block}
\begin{block}{More general forms of the chain rule}
    In how much generality can we prove that $\qd \varphi(a) \equiv \varphi'(a)\qd a$?
\end{block}
\begin{block}{Higher dimensions}
    Can we prove analogues of these results for functions on $\Rl^d$
    or $\Circ^d$?
\end{block}
\end{frame}

\begin{frame}
\frametitle{Better ideal membership results}
It can be shown that for an operator $T$, that $T \in \mathcal{M}_{1,\infty}$
if and only if 
\begin{equation*}
    \sup_{s \in (0,1)} s\|T\|_{s+1} < \infty
\end{equation*}
where
\begin{equation*}
    \|T\|_p = \left(\sum_{n\geq 0} \mu_n(T)^p\right)^{1/p}.
\end{equation*}
Since we know conditions on $\qd f$ such that $\|\qd f\|_p$ is finite,
 we (with some effort) can find conditions such that $\qd f \in \mathcal{M}_{1,\infty}$.
\end{frame}

\begin{frame}
\frametitle{More general forms of the chain rule}
Further generalisations of the chain rule can be found.

In particular it can be proved in the setting of certain \emph{spectral triples}.
\end{frame}

\begin{frame}
\frametitle{Higher dimensions}
It would appear that there an immediate generalisation of these results
in higher dimensions fails to work.

In particular, even for polynomial functions $f$, $\qd f$ is not even
$\mathcal{L}^2$.
\end{frame}

\begin{frame}
\frametitle{The End}
Thank you for listening!
\end{frame}
\end{document}
