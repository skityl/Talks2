\documentclass{beamer}
\usetheme{Madrid} % My favorite!
%\usetheme{Boadilla}
% Pretty neat, soft color.
%\usetheme{default}
%\usetheme{Warsaw}
%\usetheme{Bergen}
% This template has nagivation on the left
%\usetheme{Frankfurt} % Similar to the default
% with an extra region at the top.
%\usecolortheme{seahorse} % Simple and clean template
%\usetheme{Darmstadt} % not so good
% Uncomment the following line if you want
% % page numbers and using Warsaw theme%
% \setbeamertemplate{footline}[page number]
%\setbeamercovered{transparent}
\setbeamercovered{invisible} % To remove the navigation symbols from
% the bottom of slides
% \setbeamertemplate{navigation symbols}{}
\usepackage{graphicx}
\usepackage{calc}

\newsavebox\CBox
\newcommand\hcancel[2][0.5pt]{%
  \ifmmode\sbox\CBox{$#2$}\else\sbox\CBox{#2}\fi%
  \makebox[0.5pt][l]{\usebox\CBox}%  
  \rule[0.8\ht\CBox-#1/2]{\wd\CBox}{#1}}
  
\makeatletter
\newenvironment<>{proofs}[1][\proofname]{%
    \par
    \def\insertproofname{#1\@addpunct{.}}%
    \usebeamertemplate{proof begin}#2}
  {\usebeamertemplate{proof end}}
\makeatother


\def\qd{\,{\mathchar'26\mkern-12mu d}}
  
  

%\usepackage{bm} % For typesetting bold math (not \mathbold)
%\logo{\includegraphics[height=0.6cm]{yourlogo.eps}} %
\title[Harmonic Analysis]{Quantised Calculus in One Variable}
\author[McDonald, E.]{McDonald, E. \\Supervisor: Sukochev, F.}
\institute[UNSW] { UNSW Australia }
%\titlegraphic{\includegraphics[width=60mm]{}}
\date{\today}
\newcommand{\Rl}{\mathbb{R}}
\newcommand{\Cplx}{\mathbb{C}}
\newcommand{\Circ}{\mathbb{T}}
\newcommand{\Itgr}{\mathbb{Z}}
\newcommand{\Ntrl}{\mathbb{N}}
\newcommand{\ha}{\boldsymbol{m}}
\newcommand{\A}{\mathcal{A}}
\newcommand{\Hilb}{\mathcal{H}}
\newcommand{\D}{\mathcal{D}}
%\newcommand{\qd}{\hcancel[0.9pt]{d}}
\newcommand{\sgn}{\operatorname{sgn}}
\newcommand{\rk}{\operatorname{rank}}
\newcommand{\BMO}{\operatorname{BMO}}
\newcommand{\VMO}{\operatorname{VMO}}
\newcommand{\Proj}{\mathbf{P}}
\newcommand{\Ban}{\mathbf{Ban}}
% \today will show current date.
% Alternatively, you can specify a date.
\begin{document}
\begin{frame}
\titlepage
\end{frame} %



\begin{frame}
\frametitle{Introduction}
The purpose of this talk is to introduce the \emph{quantised calculus},
which is a tool coming from non-commutative geometry
which gives rigorous justification to computations involving ``infinitesimals".
\end{frame}

\begin{frame} 
\frametitle{Infinitesimals} 
\begin{itemize}
    \item{} Early calculus (e.g. Leibniz, Newton) made use of infinitesimal quantities:
\end{itemize}
\begin{definition}
    A quantity $x$ is called \emph{infinitesimal} if for any integer $n > 0$,
    \begin{equation*}
        |x| < \frac{1}{n}.
    \end{equation*}
\end{definition}
\begin{definition}
    A quantity $x$ is \emph{infinite} if for any integer $n > 0$, 
    \begin{equation*}
        |x| > n.
    \end{equation*}
\end{definition}
\end{frame} 

\begin{frame} 
\frametitle{Infinitesimals}
This definition was good enough for $18$th century calculus, and many
definitions were formulated in terms of infinitesimals.

For example:
\begin{definition}
    A function $f$ is continuous if $df(x) := f(x+dx)-f(x)$ is infinitesimal
    for any infinitesimal quantity $dx$.
\end{definition}
\begin{definition}
    A function $f$ is differentiable if the quantity
    \begin{equation*}
        f'(x) := \frac{f(x+dx)-f(x)}{dx}
    \end{equation*}
    is not infinite when $dx$ is infinitesimal.
\end{definition}
%\includegraphics[width=90mm]{Infinitesimal_Calculus_6.png}
\end{frame}

\begin{frame}
\frametitle{Properties of infinitesimals}
\begin{lemma}
    If $x$ is infinitesmal, then $x^{-1}$ is infinite.
\end{lemma} 
\begin{lemma}
    For any $f$,
    \begin{equation*}
        df = \frac{df}{dx}dx
    \end{equation*}
\end{lemma}
\end{frame}

\begin{frame}
\frametitle{Sizes of infinitesimals}
Not all infinitesimals are equal.
Intuitively we expect the following properties:
\begin{itemize}
\item{} If $x > 0$ is infinitesimal, than $x^2 < x$.
\item{} If a function $f$ is smoother than a function $g$, then $df < dg$.
\end{itemize}
\end{frame}

\begin{frame}
\frametitle{Problem}
There is a problem! These definitions make no sense.
\begin{block}
{Problems:}
\begin{itemize}
    \item{} If $x \neq 0$ is infinitesimal, then $|x| = 0$.

    \item{} If $x$ is infinite, then $|x| > \lfloor2|x|\rfloor$, so $1 > 2$.
\end{itemize}
\end{block}

So we cannot use the definitions given by $18$th century mathematicians.
\end{frame}


\begin{frame}
    Infinitesimals were rightly banished from mathematics, and the definitions
    of continuity and differentiability were replaced with their modern 
    definitions in terms of limits.
\end{frame}

\begin{frame}
\frametitle{But what \emph{is} an infinitesimal?}
$18$th century mathematicians were still able to use infinitesimals
even though they make no sense. 
\begin{block}
    {Question:}
    Why does calculus using infinitesmals ``work"?
\end{block}
\begin{block}
    {Question:}
    Is there a way to make sense of infinitesimals?
\end{block}
\end{frame}


\begin{frame}
\frametitle{Infinitesimal Operators}
An infinitesimal operator $T \in \mathcal{B}(\Hilb)$ should be an operator
$T$ such that for any $\varepsilon > 0$, we have
\begin{equation*}
    \|T\| < \varepsilon.
\end{equation*}

Again, this definition is useless as it implies that $T = 0$.

\end{frame}


\begin{frame}
\frametitle{Compact Operators as Infinitesimals}
We shall say that an operator $T \in \mathcal{B}(\Hilb)$
is \emph{infinitesimal} if for any $\varepsilon > 0$
there exists a finite dimensional subspace $E$
such that
\begin{equation*}
    \|T|_{E^\perp}\| < \varepsilon.
\end{equation*}
This is equivalent to saying that $T$ is compact.
\end{frame}

\begin{frame}
\frametitle{Infinitesimals in Non-commutative geometry}
In a noncommutative geometry $(\A,\Hilb,\D)$, the compact
elements of $\A$ play a similar role to infinitesimals
in classical real analysis.

We have the following dictionary:\\

\begin{tabular}{l | r}
Classical Analysis & Non-commutative Analysis\\
\hline
Function & Operator\\
One-form & Connes Differential\\
Range & Spectrum\\
Infinitesimal & Compact Operator\\
    
\end{tabular}
\end{frame}

\begin{frame}
\frametitle{Quantised Differentials}
In noncommutative geometry, we have a new object
that is not present in classical analysis called a quantised
derivative or quantised differential.
\begin{definition}
    Let $(\A,\Hilb,\D)$ be a spectral triple. Let $F$
    be the unique operator such that $\D = F|\D|$, that
    is $F = \sgn(\D)$. For $a \in \A$, define
    \begin{equation*}
        \qd a := [F,a].
    \end{equation*}
    This is the \emph{quantised differential} of $a$.
\end{definition}
$\qd a$ is supposed to represent an infinitesimal variation
of $a$. The motivation behind this definition is not clear, but as
we work through examples it will become clear that this definition
is the ``correct" one.
\end{frame} 


\begin{frame}
\frametitle{Expected properties of infinitesimals}
In classical $17$-$19$th century analysis, infinitesimals
were supposed to have a number of properties:
\begin{enumerate}
    \item{} If $f:\Rl\rightarrow \Rl$ is a function, there
    is a function $df$ representing
    infinitesimal variation in $f$. $f$ is continuous
    if and only if $df$ is infinitesimal.
    \item{} If $f$ is smoother than $g$, then $df$ is smaller
    than $dg$.
    \item{} If $x$ is a positive infinitesimal, then $x^2$
    is smaller than $x$.
    \item{} If $f$ is a differentiable function, then $df = f' dx$,
    provided that sufficiently small infinitesimals
    are ignored. 
\end{enumerate}
We shall see that the quantised differential satisfies
all these properties, if they are interpreted correctly.
\end{frame}

\begin{frame}
\frametitle{Sizes of Infinitesimals}
\begin{definition}
    Given $T \in \mathcal{B}(\Hilb)$, define the $k$th
    singular value of $T$ to be
    \begin{equation*}
        \mu_k(T) := \inf\{\|T-A\|\;:\;\rk(A) \leq k\}.
    \end{equation*}
\end{definition}
For a compact operator, $\{\mu_k(T)\}_{k=0}^\infty$ is a
vanishing sequence of positive numbers. We
shall describe the \emph{size} of an infinitesimal
$T$ as the \emph{rate of decay} of $\{\mu_k(T)\}_{k=0}^\infty$.

\end{frame}

\begin{frame}
\frametitle{Sizes of Infinitesimals}
\begin{theorem}
    Let $T$ and $S$ be compact operators in $\mathcal{B}(\Hilb)$. Then
    for every $k \geq 0$,
    \begin{equation*}
        \mu_k(T^2) \leq \mu_k(T)^2.
    \end{equation*}
\end{theorem}
Hence, if $T$ is an infinitesimal, then $T^2$ is a smaller infinitesimal.
\end{frame}

\begin{frame}
\frametitle{Sizes of Infinitesimals}
We can quantify the sizes an infinitesimal $T$ by placing conditions
on the rate of decay of $\{\mu_k(T)\}_{k=0}^\infty$.
\begin{itemize}
    \item{} The smallest infinitesimals have $\{\mu_k(T)\}_{k=0}^\infty$
    of finite support. Then $T$ is of finite rank.
    \item{} We say that $T \in \mathcal{L}^p$ if $\{\mu_k(T)\}_{k=0}^\infty \in \ell^p$.
    \item{} We say that $T \in \mathcal{L}^{p,\infty}$ if $\mu_k(T) = \mathcal{O}(k^{-1/p})$.
    \item{} We say that $T \in \mathcal{L}^{p,q}$ if $\{k^{1/p-1/q}\mu_k(T)\}_{k=0}^\infty \in \ell^q$.
    \item{} We say that $T \in \mathcal{M}^{1,\infty}$ if $\{\frac{1}{\log(k+1)}\sum_{n=0}^k \mu_k(T)\}_{k=0}^\infty \in \ell^\infty$.
\end{itemize}
\end{frame}



\begin{frame}
\frametitle{Smoothness and rate of decay}
We now enter the second phase of the talk: ``Hard Analysis".

Recall that in classical analysis, we had the statement
that if $f$ is smoother than $g$, then $df$ is smaller than $dg$.
\begin{block}
{The main question of this talk:}
In what sense is it true that if $f$ is smoother than $g$,
then $\qd f$ is smaller than $\qd g$?
\end{block}
We shall restrict attention to functions on the circle $\Circ$
and the line $\Rl$.
\end{frame}


\begin{frame}
\frametitle{Differentials on $\Circ$}

 The Dirac
operator on the circle is $\D = \frac{1}{i}\frac{d}{d\theta}$, i.e.,
$\D(z^n) = nz^n$. So we may describe $F = \sgn\D$ as the \emph{Hilbert transform}, 
\begin{equation*}
    F\left(\sum_{n\in \Itgr} a_nz^n\right) = \sum_{n\in \Itgr} \sgn(n)a_nz^n.
\end{equation*}

\end{frame}




\begin{frame}
\frametitle{Differentials on $\Circ$}
Hence for $f \in L^2(\Circ)$, 
\begin{equation*}
    Ff = \varphi * f
\end{equation*} 
where
\begin{equation*}
    \varphi = \sum_{n\in\Itgr} \sgn(n)z^n = \frac{1}{1-z}-\frac{z^{-1}}{1-z^{-1}} = \frac{2}{1-z}.
\end{equation*}

   Thus, 
\begin{align*}
    (\qd f)g &= ([F,f]g)(t) \\
    &= 2\lim_{\varepsilon \to 0} \int_{|\tau-t| > \varepsilon} \frac{f(t)-f(\tau)}{t-\tau}g(\tau)\;d\ha(\tau).
\end{align*} 
\end{frame}


\begin{frame}
\frametitle{Finite rank differentials}
Let $f:\Circ\rightarrow \Cplx$. The strictest
condition we can put on the smoothness
of $f$ is that $f$ is a rational function.
The strictest condition we can put on the size of $\qd f$
is that $\qd f$ is finite rank.
These two conditions are equivalent.
\begin{theorem}[Kronecker]
    If $f:\Circ\rightarrow \Cplx$, then $\qd f$ is finite rank
    if and only if $f$ is a rational function.
\end{theorem}   
\end{frame}

\begin{frame}
\frametitle{Bounded differentials}
Let $f:\Circ\rightarrow \Cplx$. The weakest condition
that we can place on $\qd f$ is that $\qd f$ is bounded.
\begin{definition}
    Let $f:\Circ\rightarrow \Cplx$ be measureable. We say that $f$ is of \emph{bounded mean
    oscillation} if for an arc $I \subseteq \Circ$, define
    \begin{equation*}
        f_I = \frac{1}{\ha(I)}\int_I f\;d\ha
    \end{equation*}
    and
    \begin{equation*}
        \sup_{I} \frac{1}{\ha(I)}\int_I |f-f_I|\;d\ha < \infty
    \end{equation*}
    where the supremum runs over all arcs $I$.
    The set of functions with bounded mean oscillation is denoted $\BMO(\Circ)$.
\end{definition}
\end{frame} 

\begin{frame}
\frametitle{Bounded differentials}
\begin{theorem}[Nehari]
    Let $f:\Circ\rightarrow \Cplx$. Then $\qd f$ is bounded
    if and only if $f \in \BMO(\Circ)$.
\end{theorem}
\end{frame}

\begin{frame}
\frametitle{Compact differentials}
We define the space $\VMO(\Circ)$:
\begin{definition}
    We say that $f \in \VMO(\Circ)$ if $f \in \BMO(\Circ)$
    and
    \begin{equation*}
        \lim_{\ha(I)\rightarrow 0} \frac{1}{\ha(I)}\int_I |f-f_I|\;d\ha = 0.
    \end{equation*}
\end{definition}
\begin{theorem}
    If $f:\Circ\rightarrow \Cplx$, then $\qd f$ is compact if and only if $f \in \VMO(\Circ)$.
\end{theorem}
\end{frame}

\begin{frame}
\frametitle{Can we do better?}
We seek a more precise characterisation of the relationship between
the smoothness
of $f$ and the size of $\qd f$. To this end, we define the \emph{Besov classes} $B^s_{pq}$.
\end{frame}

\begin{frame}
\frametitle{Definition of $B_{pq}^s$} 
We define a sequence of polynomials $\{W_n\}_{n\in \Itgr}$ on
$\Circ$ as follows.
\begin{enumerate}
    \item{} $W_0 = z^{-1}+1+z$.
    \item{} For $n > 0$ and $k > 0$,
    \begin{equation*}
        \widehat{W}_n(k) = \begin{cases}
            1,\text{ if }k = 2^n\\
            \text{a linear function on }[2^{n-1},2^n]\text{ and }[2^n,2^{n+1}]\\
            0,\text{ otherwise.}
        \end{cases}
    \end{equation*}
    and $\widehat{W}_n(-k) = \widehat{W}_n(k)$, and $\widehat{W}_n(0) = 0$.
\end{enumerate}
\end{frame}

\begin{frame}
\frametitle{Definition of $B^s_{pq}$}
\begin{definition}
    For an integrable function $\varphi$ on $\Circ$, we say that $f \in B_{pq}^s(\Circ)$
    if
    \begin{equation*}
        \sum_{n\in \Itgr} 2^{|n|sp} \|W_n*\varphi\|_{p}^q < \infty.
    \end{equation*}
\end{definition}
\end{frame}

\begin{frame}
\frametitle{Quantised differentials of Schatten-Von Neumann class}
\begin{theorem}[Peller]
    Let $f$ be a measurable function on $\Circ$
    and $p > 0$. Then $\qd f \in \mathcal{L}^p$ if and only
    if $f \in B_{pp}^{1/p}(\Circ)$.
\end{theorem}
\end{frame}

\begin{frame}
\frametitle{The End}
In summary:
\begin{enumerate}
    \item{} The quantised differential has meaning in analysis on $\Circ$
    and $\Rl$. 
    \item{} The quantised differential has many properties that we would 
    anticipate for ``infinitesimal differences".
\end{enumerate}
Thank you for listening!
\end{frame}
\end{document}
