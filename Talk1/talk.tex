\documentclass{beamer}
\usetheme{Madrid} % My favorite!
%\usetheme{Boadilla}
% Pretty neat, soft color.
%\usetheme{default}
%\usetheme{Warsaw}
%\usetheme{Bergen}
% This template has nagivation on the left
%\usetheme{Frankfurt} % Similar to the default
% with an extra region at the top.
%\usecolortheme{seahorse} % Simple and clean template
%\usetheme{Darmstadt} % not so good
% Uncomment the following line if you want
% % page numbers and using Warsaw theme%
% \setbeamertemplate{footline}[page number]
%\setbeamercovered{transparent}
\setbeamercovered{invisible} % To remove the navigation symbols from
% the bottom of slides
% \setbeamertemplate{navigation symbols}{}
\usepackage{graphicx}
%\usepackage{bm} % For typesetting bold math (not \mathbold)
%\logo{\includegraphics[height=0.6cm]{yourlogo.eps}} %
\title[Harmonic Analysis]{A Gentle Introduction to Non-commutative Harmonic Analysis}
\author[McDonald, E.]{McDonald, E. S. \\Supervisor: Sukochev, F.}
\institute[UNSW] { UNSW Australia }
%\titlegraphic{\includegraphics[width=60mm]{}
\date{\today}
\newcommand{\Rl}{\mathbb{R}}
\newcommand{\Cplx}{\mathbb{C}}
\newcommand{\Circ}{\mathbb{T}}
\newcommand{\Itgr}{\mathbb{Z}}
\newcommand{\ha}{\boldsymbol{m}}
% \today will show current date.
% Alternatively, you can specify a date.
\begin{document}
\begin{frame}
\titlepage
\end{frame} %
\begin{frame}
\frametitle{Introduction}
    \begin{block}
            {Question for this lecture:}
                How do Fourier Series work?
        \end{block}
    \begin{block}
            {Another Question:}
                What is the non-commutative torus?
    \end{block}
\end{frame}

\begin{frame}
    \frametitle{Fourier Analysis In Kindergarten}
    \begin{itemize}
        \item{} In second year we learn the following situation:
    \end{itemize}
    \begin{definition}
        If $f:\Rl\rightarrow\Cplx$ is a function that is periodic with period $P$, then
        \begin{equation*}
            f(x) = \frac{a_0}{2}+\sum_{n=1}^\infty \left(a_n \cos\left(\frac{2\pi n x}{P}\right)+b_n\sin\left(\frac{2\pi n x}{P}\right)\right)
        \end{equation*}
    \end{definition}
    \begin{definition}
        \begin{align*}
            a_n &= \frac{2}{P}\int_{0}^P f(x) \cos\left(\frac{2\pi n x}{P}\right)\;dx\\
            b_n &= \frac{2}{P}\int_{0}^P f(x) \sin\left(\frac{2\pi n x}{P}\right)\;dx
        \end{align*}
    \end{definition}
\end{frame}

\begin{frame}
    \frametitle{Fourier Analysis In Kindergarten}
        We normally describe the coefficients $\{a_n,b_n\}_{n=0}^\infty$
        as the ``Fourier Transform" of $f$.
        
    
        There are many questions left unanswered...
        \begin{itemize}
            \item{} Does this really work for \emph{any} function $f$?
            \item{} In what sense is the Fourier series meant to converge?
        \end{itemize}
\end{frame}

\begin{frame}
    \frametitle{A better notation}
        The notation that was used in second year,
        \begin{equation*}
            f(x) = \frac{a_0}{2}+\sum_{n=1}^\infty \left(a_n \cos\left(\frac{2\pi n x}{P}\right)+b_n\sin\left(\frac{2\pi n x}{P}\right)\right),
        \end{equation*}
        is rather silly.
        This;
        \begin{equation*}
            f(x) = \sum_{n=-\infty}^\infty c_n \exp\left(\frac{2\pi in x}{P}\right)
        \end{equation*}
        is much better.
\end{frame}

\begin{frame}
    \frametitle{An even better notation still}
    Let
    \begin{equation*}
        \Circ = \{\zeta \in \Cplx\;:\;|\zeta|=1\}.
    \end{equation*}
    We can identify periodic functions on $\Rl$
    with functions on $\Circ$. 
    Let
    \begin{equation*}
        z:\Circ\rightarrow\Circ
    \end{equation*}
    be the identity function.
    If we are identifying $P$-periodic functions on $\Rl$
    with functions on $\Circ$, then we identify
    \begin{equation*}
        \exp\left(\frac{2\pi i n x}{P}\right) = z^n.
    \end{equation*}
\end{frame}

\begin{frame}
    \frametitle{Integrating functions on $\Circ$}
    $\Circ$ has a measure: normalised arc length.
    We call this measure $\ha$, so that
    if $f:\Circ\rightarrow\Cplx$,
    \begin{equation*}
        \int_\Circ f\;d\ha = \int_{0}^1 f\left(\exp(2\pi i t)\right)\;dt.
    \end{equation*}
\end{frame}

\begin{frame}
    \frametitle{An even better notation still}
    So with our new notation, Fourier series look like this:
    \begin{definition}
        If $f:\Circ\rightarrow \Cplx$, then the fourier series
        of $f$ is
        \begin{equation*}
            \sum_{n \in \Itgr} c_n z^n.
        \end{equation*}
        where
        \begin{equation*}
            c_n = \int_\Circ z^{-n} f\;d\ha.
        \end{equation*}
    \end{definition}
\end{frame}

\begin{frame}
    \frametitle{Fourier Series for Grown-ups}
    For $n \in \Itgr$, and $f \in L^1(\Circ,\ha)$,
    let
    \begin{equation*}
        \widehat{f}(n) := \int_\Circ z^{-n}f\;d\ha.
    \end{equation*}
    So we have a ``correspondence",
    \begin{equation*}
        f \sim \sum_{n \in \Itgr} \widehat{f}(n) z^n.        
    \end{equation*}
    What does ``$\sim$" mean?
\end{frame}

\begin{frame}
    \frametitle{What does the Fourier Series mean?}
    At the moment when we write
    \begin{equation*}
        f \sim \sum_{n \in \Itgr} \widehat{f}(n) z^n
    \end{equation*}
    the right hand side is just a formal power series. What do
    we have to do to make the ``$\sim$" into an ``$=$"?
    
    How shall the sum on the right hand side be interpreted? What
    kind of object is this?
\end{frame}

\begin{frame}
    \frametitle{Divergent Series}
    Normally in mathematics, when we write
    \begin{equation*}
        \sum_{n=0}^\infty a_n,
    \end{equation*}
    we mean
    \begin{equation*}
        \lim_{N\rightarrow\infty}\sum_{n=0}^N a_n.
    \end{equation*}
    This means that series like
    \begin{align*}
        &\sum_{n=0}^\infty (-1)^n\\
        &\sum_{n=0}^\infty n
    \end{align*}
    do not make sense. This is ``classical summation".
\end{frame}

\begin{frame}
    \frametitle{Divergent Series}
    It is often desirable to assign a ``sensible" value
    to a series that is divergent. This is called \emph{regularisation}.
    
    
    What this means is redefining what we mean by
    \begin{equation*}
        \sum_{n=0}^\infty a_n.
    \end{equation*}
\end{frame}

\begin{frame}
    \frametitle{Abel Summation}
    Abel Summation is a method of regularisation that is motivated
    by \emph{Abel's theorem}.
    \begin{theorem}
        Let $f(x) = \sum_{n=0}^\infty a_n x^n$ be a power
        series that converges in $(-1,1)$. Suppose
        that
        \begin{equation*}
            \sum_{n=0}^\infty a_n
        \end{equation*}
        exists. Then
        \begin{equation*}
            \sum_{n=0}^\infty a_n = \lim_{x\rightarrow 1^{-}} \sum_{n=0}^\infty a_nx^n.
        \end{equation*}
    \end{theorem}
\end{frame}

\begin{frame}
    \frametitle{Abel summation}
    \begin{definition}
        Let $\{a_n\}_{n=0}^\infty$ be a sequence, the \emph{Abel sum}
        of $\{a_n\}_{n=0}^\infty$ is
        \begin{equation*}
            _{(A)}\sum_{n=0}^\infty a_n := \lim_{r\rightarrow 1^-} \sum_{n=0}^\infty a_n r^n.
        \end{equation*}
    \end{definition}
\end{frame}

\begin{frame}
    \frametitle{Abel summation}
    For example,
    \begin{align*}
        _{(A)}\sum_{n=0}^\infty (-1)^n &= \lim_{r\rightarrow 1^-} \sum_{n=0}^\infty (-r)^n \\
                                       &= \lim_{r\rightarrow 1^-} \frac{1}{1+r}\\
                                       &= \frac{1}{2}.
    \end{align*}
    Abel's theorem guarantees that if a series is summable in the classical
    sense, then it is Abel summable and the Abel and classical sums agree.
\end{frame}

\begin{frame}
    \frametitle{Aside}
    Using a completely different method of regularisation, we
    can give rigorous justification to the (infamous) expression,
    \begin{equation*}
        1+2+3+4+\cdots = -\frac{1}{12}.
    \end{equation*}
\end{frame}

\begin{frame}
    \frametitle{Back to Fourier Series}
    Let $f \in L^1(\Circ)$, define
    \begin{equation*}
        A_r f := \sum_{n \in \Itgr} \widehat{f}(n) r^nz^n.
    \end{equation*}
    This exists for any $r$.
    
    The question of whether a Fourier series is Abel summable
    is equivalent to considering the limit $\lim_{r\rightarrow 1^-} A_rf$.
\end{frame}

\begin{frame}
    \frametitle{Back to Fourier Series}
    The following theorem is remarkably easy to prove:
    \begin{theorem}
        Suppose that $f \in L^p(\Circ,\ha)$, with $p \in [1,\infty)$. 
        Then
        \begin{equation*}
            f = \lim_{r\rightarrow 1^-} A_r f = _{(A)}\sum_{n \in \Itgr} \widehat{f}(n)z^n
        \end{equation*}
        where the limit is in the $L^p$ sense.
    \end{theorem}
\end{frame}

\begin{frame}
    \frametitle{The final word on Fourier Series?}
    So at last we have rigorous meaning for the expression,
    \begin{equation*}
        f = \sum_{n\in \Itgr}\widehat{f}(n)z^n.
    \end{equation*}
    We simply need to interpret the right hand side as an Abel sum,
    converging in the $L^p$ sense, if $f \in L^p(\Circ,\ha)$ and $p \in [1,\infty)$.
\end{frame}

\begin{frame}
    \frametitle{Can we do better?}
    Abel summation is nice, but what really interests
    us is classical summation. Carleson and Hunt proved the following,
    \begin{theorem}
        Suppose that $f \in L^p(\Circ,\ha)$, for $p \in (1,\infty)$, then
        \begin{equation*}
            f = \sum_{n\in\Itgr} \widehat{f}(n)z^n
        \end{equation*}
        where the sum is a classical sum, converging in the $L^p$ sense.
    \end{theorem}
\end{frame} 

\begin{frame}
    \frametitle{What about the $p = 1$ case?}
    The theorem spectacularly fails to be true when $p=1$.
    The following is a result of Kolmogorov,
    \begin{theorem}
        There is a function $k \in L^1(\Circ,\ha)$ such that
        \begin{equation*}
            \zeta \mapsto \sum_{n\in\Itgr} \widehat{k}(n)\zeta^n
        \end{equation*}
        diverges for every $\zeta \in \Circ$.
    \end{theorem}
\end{frame}

\begin{frame}
    \frametitle{What about higher dimensions?}
    Let $f \in L^1(\Circ^2,\ha\times\ha)$. 
    Let $z,w:\Circ^2\rightarrow\Circ$ be the first and second coordinate functions.
    
    Then we can write,
    \begin{equation*}
        f = \lim_{r,s\rightarrow 1^-} \sum_{(n,m) \in \Itgr^2} \widehat{f}(n,m)r^ns^mz^nw^m.
    \end{equation*}
    where
    \begin{equation*}
        \widehat{f}(n,m) = \int_{\Circ^2} z^{-n}w^{-m} f\;d(\ha\times\ha).
    \end{equation*}
\end{frame}

\begin{frame}
    \frametitle{What about higher dimensions?}
    In other words, most functions on $\Circ^2$ can be written as a doubly-indexed
    power series,
    \begin{equation*}
        f = \sum_{(n,m) \in \Itgr^2} c_{n,m} z^nw^m
    \end{equation*}
    for some  coefficients $c_{n,m}$, and the series converges
    in ``some sense".
    
    
    In particular, the subspace
    \begin{equation*}
        \left\{ \sum_{(n,m) \in \Itgr^2} c_{n,m}z^nw^m\;:\; \sup_{n,m} (1+|n|+|m|)^k|c_{n,m}| < \infty \text{ for all }k \geq 0\right\}.
    \end{equation*}
    is exactly $C^\infty(\Circ^2)$.
\end{frame}


\begin{frame}
    \frametitle{Non-commutative Geometry}
    Question: What is non-commutative geometry?
    
    \begin{block}
        {Answer:}   
            The study of non-commutative algebras which are somehow similar
            to algebras of functions on geometric spaces, using the methods
            and language of geometry.
    \end{block}
\end{frame}

\begin{frame}
    \frametitle{Introducing the non-commutative torus}
    Let $\theta$ be an irrational number.
    \begin{definition}
        The non-commutatative torus $\mathcal{A}_\theta$ is a $\Cplx$-$*$-algebra
        generated by two elements $U$ and $V$ satisfying the commutation relation
        \begin{equation*}
            UV = e^{2\pi i \theta} VU.
        \end{equation*}
        and $U^{-1} = U^*$,$V^{-1} = V^*$
        and every element of $\mathcal{A}_\theta$ can be written as
        \begin{equation*}
            \sum_{(n,m) \in \Itgr^2} c_{n,m} U^n V^m.
        \end{equation*}
        where $c_{n,m}$ is a sequence satisfying $\sup_{n,m} (1+|n|+|m|)^k |c_{n,m}| < \infty$.
    \end{definition}
\end{frame}

\begin{frame}
    \frametitle{The Upshot}
    The algebra $\mathcal{A}(\Circ^2_\theta)$ is a lot
    like an algebra of functions on $\Circ^2$,
    and many techniques of harmonic analysis can be used
    on $\mathcal{A}(\Circ^2_\theta)$. So this
    justifies the term ``non-commutative harmonic analysis".
\end{frame}

\begin{frame}
    The end!
\end{frame}

\end{document}

