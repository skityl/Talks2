\documentclass{beamer}
\usetheme{Madrid} % My favorite!
%\usetheme{Boadilla}
% Pretty neat, soft color.
%\usetheme{default}
%\usetheme{Warsaw}
%\usetheme{Bergen}
% This template has nagivation on the left
%\usetheme{Frankfurt} % Similar to the default
% with an extra region at the top.
%\usecolortheme{seahorse} % Simple and clean template
%\usetheme{Darmstadt} % not so good
% Uncomment the following line if you want
% % page numbers and using Warsaw theme%
% \setbeamertemplate{footline}[page number]
%\setbeamercovered{transparent}
\setbeamercovered{invisible} % To remove the navigation symbols from
% the bottom of slides
% \setbeamertemplate{navigation symbols}{}
\usepackage{graphicx}
\usepackage{calc}

\newsavebox\CBox
\newcommand\hcancel[2][0.5pt]{%
  \ifmmode\sbox\CBox{$#2$}\else\sbox\CBox{#2}\fi%
  \makebox[0.5pt][l]{\usebox\CBox}%  
  \rule[0.8\ht\CBox-#1/2]{\wd\CBox}{#1}}
  
\makeatletter
\newenvironment<>{proofs}[1][\proofname]{%
    \par
    \def\insertproofname{#1\@addpunct{.}}%
    \usebeamertemplate{proof begin}#2}
  {\usebeamertemplate{proof end}}
\makeatother

  
  
  

%\usepackage{bm} % For typesetting bold math (not \mathbold)
%\logo{\includegraphics[height=0.6cm]{yourlogo.eps}} %
\title[Harmonic Analysis]{Quantised Calculus and Hankel Operators}
\author[McDonald, E.]{McDonald, E. \\Supervisor: Sukochev, F.}
\institute[UNSW] { UNSW Australia }
%\titlegraphic{\includegraphics[width=60mm]{}
\date{\today}
\newcommand{\Rl}{\mathbb{R}}
\newcommand{\Cplx}{\mathbb{C}}
\newcommand{\Circ}{\mathbb{T}}
\newcommand{\Itgr}{\mathbb{Z}}
\newcommand{\Ntrl}{\mathbb{N}}
\newcommand{\ha}{\boldsymbol{m}}
\newcommand{\A}{\mathcal{A}}
\newcommand{\Hilb}{\mathcal{H}}
\newcommand{\D}{\mathcal{D}}
\newcommand{\qd}{\hcancel[0.9pt]{d}}
\newcommand{\sgn}{\operatorname{sgn}}
\newcommand{\rk}{\operatorname{rank}}
\newcommand{\BMO}{\operatorname{BMO}}
\newcommand{\VMO}{\operatorname{VMO}}
\newcommand{\Proj}{\mathbf{P}}
\newcommand{\Ban}{\mathbf{Ban}}
% \today will show current date.
% Alternatively, you can specify a date.
\begin{document}
\begin{frame}
\titlepage
\end{frame} %



\begin{frame}
\frametitle{Introduction}
The purpose of this talk is to introduce the \emph{quantised calculus},
which is a tool coming from non-commutative geometry
which gives rigorous justification to computations involving ``infinitesimals".

We shall introduce noncommutative geometry, define the quantised calculus, and 
give an in-depth discussion of quantised calculus on the circle.
\end{frame}


\begin{frame}
\frametitle{Introduction}
    This talk will be divided into two parts:
    \begin{block}
            {Soft Analysis:}
                The motivation for quantised calculus from non-commutative geometry.
        \end{block}
    \begin{block}
            {Hard Analysis:}
                Operator ideal membership of quantised derivatives.
    \end{block}
\end{frame}

\begin{frame}
\frametitle{Non-commutative Geometry}
\begin{block}
    {What is Non-commutative geometry?}
        The study of noncommutative algebras which are similar to algebras
        of functions on geometric spaces, using the methods
        and language of geometry.
\end{block}
\end{frame}

\begin{frame}
\frametitle{Why Non-commutative Geometry?}
Non-commutative geometry gives us new insights on non-commutative algebra,
but historically it was motivated by quantum mechanics.
\end{frame}

\begin{frame}
\frametitle{A very brief description of quantum mechanics}
\begin{definition}
    A \emph{quantum mechanical system} is a pair $(\A,\Hilb)$ where
    $\Hilb$ is a complex separable Hilbert space
    and $\A$ is a $*$-algebra of (potentially unbounded, densely defined)
    operators on $\Hilb$. The self adjoint elements of $\A$
    are called \emph{observables}. The non-zero elements of $\Hilb$ are called
    \emph{states}.
\end{definition}
\begin{itemize}
\item{} The states correspond to potential configurations of a physical system.
States which are non-zero scalar multiples of each other are considered physically indistinguishable.

\item{} The observables correspond to physical quantities that can be measured.
The range of potential measurements for an observable $A \in \A$
is $\sigma(A)$.
\end{itemize}
\end{frame}

\begin{frame}
\frametitle{Reminder: The spectral theorem}
\begin{theorem}
    Let $(\A,\Hilb)$ be a quantum mechanical system. If $A \in \A$
    is an observable, then there exists a projection
    valued measure $E_A$ on $\sigma(A)$ such that
    \begin{equation*}
            A = \int_{\sigma(A)} \lambda\;dE_A(\lambda).
    \end{equation*}
\end{theorem}
\end{frame}

\begin{frame}
\frametitle{Measurement in quantum mechanics}
Let $(\A,\Hilb)$ be a quantum mechanical system, currently in a state $\psi \in \Hilb$. 

Let $A \in \A$ be an observable. The probability that the measured
value of the physical quantity corresponding to $A$ lies in 
a borel set $\Delta \subseteq \sigma(A)$ is
\begin{equation*}
    P(\Delta;\psi) := \frac{(\psi,E_A(\Delta)\psi)}{\|\psi\|^2} = \frac{\|E_A(\Delta)\psi\|^2}{\|\psi\|^2}.
\end{equation*}

If the measured value lies in $\Delta$, then the state of the system changes to
\begin{equation*}
    {E_A(\Delta)\psi}.
\end{equation*}
\end{frame}

\begin{frame}
\frametitle{Measurement in quantum mechanics}
There are two surprising features of measurement in quantum mechanics:
\begin{enumerate}
    \item{} The act of observation changes the state of the system.
    \item{} The order of observation is important, since if $A$ and $B$
    are observables, and $\Delta \times \Sigma \subseteq \sigma(A)\times\sigma(B)$, then
    in general it is not true that
    \begin{equation*}
        E_A(\Delta)E_B(\Sigma)\psi = E_B(\Sigma)E_A(\Delta)\psi.
    \end{equation*}
\end{enumerate}
\end{frame}

\begin{frame}
\frametitle{The beginnings of noncommutative geometry}
A noncommutative space is a ``space" with ``coordinates"
such that the order of measurement of coordinates
changes the measured values. 

Inspired by quantum mechanics, we define a noncommutative space
as a pair $(\A,\Hilb)$ exactly like a quantum mechanical system.
\end{frame}

\begin{frame}
\frametitle{Relation to Ordinary Geometry}
The usual setting for geometry is a Riemannian manifold $M$. Let $\A = C^\infty(M)$
and $\Hilb = L^2(M)$.

However, we need more information to specify the geometry of the manifold. One way
to do this is by giving a \emph{Dirac operator}.
\end{frame}

\begin{frame}
\frametitle{Dirac Operators}
Associated to a Riemannian manifold $(M,g)$ is a \emph{clifford bundle}.
\begin{equation*}
    \operatorname{Cliff}(M,g) = \frac{T(TM)}{\langle xy+yx = -2g(x,y)\rangle }.
\end{equation*}
A vector bundle $V$ on $M$ is called a \emph{clifford module} if there
is a left multiplication action $m:\operatorname{Cliff}(M,g)\otimes V\rightarrow V$.
\end{frame}

\begin{frame}
\frametitle{Dirac Operators}
\begin{definition}
    Suppose that $(M,g)$ is a Riemannian manifold with a Clifford module 
    $V$, and a $V$-valued connection $\nabla:V\rightarrow T^*M\otimes V$. 
    Then the \emph{Dirac Operator} on $V$ is given by the composition of maps
    \begin{equation*}
            V \xrightarrow{\nabla} T^*M \otimes V \xrightarrow{\sharp\otimes I} TM\otimes V \xrightarrow{m} V
    \end{equation*}
    where $\sharp$ is the canonical isomorphism from $T^*M$ to $TM$ given by $g$.
\end{definition}
\end{frame}

\begin{frame}
    \begin{theorem}
        Let $(M,g)$ be a Riemannian manifold with Clifford module $V$
        and Dirac operator $\D$. Then for every $a \in C^\infty(M)$,
        \begin{equation*}
            da = [\D,a]
        \end{equation*}
        as an equality of operators on the sections of $V$.
    \end{theorem}
\end{frame}

\begin{frame}
\frametitle{Spectral Triples}
To specify the geometry of non-commutative space $(\A,\Hilb)$, we need
additional data to tell us how observables relate to each other. An elegant
way of doing this is by giving a Dirac operator.
\begin{definition}
    A \emph{spectral triple} is a triple $(\A,\Hilb,\D)$ where $\Hilb$ is a complex
    separable Hilbert space and $\A$ is a $*$-algebra of bounded operators on $\Hilb$. 
    $\D$ is a densely defined unbounded self adjoint operator on $\Hilb$ such that:
    \begin{enumerate}
        \item{} $[\D,a]$ is bounded for all $a \in \A$. 
        \item{} $(\lambda-\D)^{-1}$ is compact for all $\lambda \in \Cplx\setminus \Rl$.
    \end{enumerate}
\end{definition}
The prototypical example is a Riemannian manifold $M$
with a Clifford module, where $\A = C^\infty(M)$, $\Hilb = L^2(V)$
and $\D$ is the Dirac operator of the module.
\end{frame}

\begin{frame}
\frametitle{Properties of Spectral Triples}
\begin{definition}
    Let $k \geq 0$ be an integer. We say that a spectral triple $(\A,\Hilb,\D)$
    is $QC^k$ if for any $a \in \A$, $a$ and $[\D,a]$ are in the domain of $\delta^k(a)$,
    where $\delta(a) := [|\D|,a]$.
\end{definition}
\begin{definition}
    Let $p > 0$. A spectral triple $(\A,\Hilb,\D)$ is called $(p,\infty)$ summable
    if $(\lambda-\D)^{-1} \in \mathcal{L}^{p,\infty}$ for $\lambda \in \Cplx\setminus \Rl$. $p$
    is analogous to the dimension of the space.
\end{definition}
\end{frame}

\begin{frame}
\frametitle{Derivations in abstract algebra}
\begin{definition}
    Let $A$ be a $\Cplx$-algebra, and $M$ is an $A$-bimodule. 
    A $\Cplx$-linear map $\theta:A\rightarrow M$ is called
    a derivation if $\theta(ab) = a\theta(b) + \theta(a)b$ for all
    $a,b \in A$. 
\end{definition}
\begin{example}
    Let $N$ be a manifold.
    We can take $A = C^\infty(N)$ and $M = \Omega^1(N)$,
    and $\theta = d$.
\end{example}
\end{frame}

\begin{frame}
\frametitle{K\"ahler Differentials}
For a $\Cplx$-algebra $\A$, there is a ``largest" $A$-bimodule that
is the image of a derivation on $\A$. 
\begin{definition}
    Let $A$ be a $\Cplx$-algebra. There is a multiplication
    map $\gamma:A\otimes A\rightarrow A$. Define 
    \begin{equation*}
        \Omega^1(A) := \ker\gamma.
    \end{equation*}
    And $d:A\rightarrow \Omega^1(A)$ is given by $a\mapsto 1\otimes a-a\otimes 1$.
    This is the algebra of \emph{K\"ahler differentials}.
\end{definition}    
\end{frame}

\begin{frame}
\frametitle{K\"ahler differentials}
The utility of K\"ahler differentials comes from the universal property:
\begin{theorem}
    Let $A$ be a $\Cplx$-algebra, and $M$ is an $A$-bimodule 
    with a derivation $\theta:A\rightarrow M$. Then there exists
    a unique $A$-module homomorphism $\tilde{\theta}:\Omega^1(A)\rightarrow M$ such that
    $\theta = \tilde{\theta}\circ d$. 
\end{theorem}
\end{frame}

\begin{frame}
\frametitle{Connes Differentials}
Given a spectral triple, $(\A,\Hilb,\D)$, and $a \in \A$, we define
\begin{equation*}
    \pi:\Omega^1(\A)\rightarrow \mathcal{B}(\Hilb)
\end{equation*}
by 
\begin{equation*}
    da \mapsto [\D,a].
\end{equation*}
Define the two-sided ideal of $\mathcal{B}(\Hilb)$, $J$ as
\begin{equation*}
    J = \{t \in \Omega^1(\A)\;:\; \pi(t) = 0\}.
\end{equation*}
and $J_0 = J+dJ$.
Then,
\begin{equation*}
    \Omega^1_\D(\A) := \frac{\pi(\Omega^1(\A))}{\pi(dJ_0)}.
\end{equation*}
\end{frame}

\begin{frame}
\frametitle{Infinitesimals}
$17$-$19$th century mathematicians made frequent use of \emph{infinitesimals}. 
An infintesimal was a quantity $x$ such that for all $\varepsilon > 0$, we have
\begin{equation*}
    |x| < \varepsilon.
\end{equation*}
Of course, this implies that $x = 0$, so the use of non-zero infinitesimals was
abolished.
\end{frame}

\begin{frame}
\frametitle{Infinitesimal Operators}
An infinitesimal operator $T \in \mathcal{B}(\Hilb)$ should be an operator
$T$ such that for any $\varepsilon > 0$, we have
\begin{equation*}
    \|T\| < \varepsilon.
\end{equation*}

Again, this definition is useless as it implies that $T = 0$.

\end{frame}


\begin{frame}
\frametitle{Compact Operators as Infinitesimals}
We shall say that an operator $T \in \mathcal{B}(\Hilb)$
is \emph{infinitesimal} if for any $\varepsilon > 0$
there exists a finite dimensional subspace $E$
such that
\begin{equation*}
    \|T|_{E^\perp}\| < \varepsilon.
\end{equation*}
This is equivalent to saying that $T$ is compact.
\end{frame}

\begin{frame}
\frametitle{Infinitesimals in Non-commutative geometry}
In a noncommutative geometry $(\A,\Hilb,\D)$, the compact
elements of $\A$ play a similar role to infinitesimals
in classical real analysis.

We have the following dictionary:\\

\begin{tabular}{l | r}
Classical Analysis & Non-commutative Analysis\\
\hline
Function & Operator\\
One-form & Connes Differential\\
Range & Spectrum\\
Infinitesimal & Compact Operator\\
    
\end{tabular}
\end{frame}

\begin{frame}
\frametitle{Quantised Differentials}
In noncommutative geometry, we have a new object
that is not present in classical analysis called a quantised
derivative or quantised differential.
\begin{definition}
    Let $(\A,\Hilb,\D)$ be a spectral triple. Let $F$
    be the unique operator such that $\D = F|\D|$, that
    is $F = \sgn(\D)$. For $a \in \A$, define
    \begin{equation*}
        \qd a := [F,a].
    \end{equation*}
    This is the \emph{quantised differential} of $a$.
\end{definition}
$\qd a$ is supposed to represent an infinitesimal variation
of $a$. The motivation behind this definition is not clear, but as
we work through examples it will become clear that this definition
is the ``correct" one.
\end{frame} 


\begin{frame}
\frametitle{Expected properties of infinitesimals}
In classical $17$-$19$th century analysis, infinitesimals
were supposed to have a number of properties:
\begin{enumerate}
    \item{} If $f:\Rl\rightarrow \Rl$ is a function, there
    is a function $df$ representing
    infinitesimal variation in $f$. $f$ is continuous
    if and only if $df$ is infinitesimal.
    \item{} If $f$ is smoother than $g$, then $df$ is smaller
    than $dg$.
    \item{} If $x$ is a positive infinitesimal, then $x^2$
    is smaller than $x$.
    \item{} If $f$ is a differentiable function, then $df = f' dx$,
    provided that sufficiently small infinitesimals
    are ignored. 
\end{enumerate}
We shall see that the quantised differential satisfies
all these properties, if they are interpreted correctly.
\end{frame}

\begin{frame}
\frametitle{Sizes of Infinitesimals}
\begin{definition}
    Given $T \in \mathcal{B}(\Hilb)$, define the $k$th
    singular value of $T$ to be
    \begin{equation*}
        \mu_k(T) := \inf\{\|T-A\|\;:\;\rk(A) \leq k\}.
    \end{equation*}
\end{definition}
For a compact operator, $\{\mu_k(T)\}_{k=0}^\infty$ is a
vanishing sequence of positive numbers. We
shall describe the \emph{size} of an infinitesimal
$T$ as the \emph{rate of decay} of $\{\mu_k(T)\}_{k=0}^\infty$.

\end{frame}

\begin{frame}
\frametitle{Sizes of Infinitesimals}
\begin{theorem}
    Let $T$ and $S$ be compact operators in $\mathcal{B}(\Hilb)$. Then
    for every $k \geq 0$,
    \begin{equation*}
        \mu_k(TS) \leq \mu_k(T)\mu_k(S).
    \end{equation*}
\end{theorem}
Hence, if $T$ is an infinitesimal, then $T^2$ is a smaller infinitesimal.
\end{frame}

\begin{frame}
\frametitle{Sizes of Infinitesimals}
We can quantify the sizes an infinitesimal $T$ by placing conditions
on the rate of decay of $\{\mu_k(T)\}_{k=0}^\infty$.
\begin{itemize}
    \item{} The smallest infinitesimals have $\{\mu_k(T)\}_{k=0}^\infty$
    of finite support. Then $T$ is of finite rank.
    \item{} We say that $T \in \mathcal{L}^p$ if $\{\mu_k(T)\}_{k=0}^\infty \in \ell^p$.
    \item{} We say that $T \in \mathcal{L}^{p,\infty}$ if $\mu_k(T) = \mathcal{O}(k^{-1/p})$.
    \item{} We say that $T \in \mathcal{L}^{p,q}$ if $\{k^{1/p-1/q}\mu_k(T)\}_{k=0}^\infty \in \ell^q$.
    \item{} We say that $T \in \mathcal{M}^{1,\infty}$ if $\{\frac{1}{\log(k+1)}\sum_{n=0}^k \mu_k(T)\}_{k=0}^\infty \in \ell^\infty$.
\end{itemize}
\end{frame}

\begin{frame}
\frametitle{An initial result about quantised differentials}
\begin{lemma}
    Suppose that $(\A,\Hilb,\D)$ is a $QC^1$, $(p,\infty)$-summable spectral triple, 
    where we assume that $\D$ is invertible. Then for
    all $a \in \A$, $\qd a \in \mathcal{L}^{p,\infty}$.
\end{lemma}
\begin{proof}
    It can be proved that $|\D|^{-1} \in \mathcal{L}^{p,\infty}$. Now we write,
    \begin{align*}
        [\D,a] &= [F|\D|,a]\\
        &= F[|\D|,a] + [F,a]|\D|.
    \end{align*}
    Hence,
    \begin{align*}
        \qd a &= [\D,a]|\D|^{-1}-F[|\D|,a]|\D|^{-1}\\
              &= (da-F\delta(a))|\D|^{-1}.
    \end{align*}
    Therefore $\qd a \in \mathcal{L}^{p,\infty}$.
\end{proof}
\end{frame}


\begin{frame}
\frametitle{Smoothness and rate of decay}
We now enter the second phase of the talk: ``Hard Analysis".

Recall that in classical analysis, we had the statement
that if $f$ is smoother than $g$, then $df$ is smaller than $dg$.
\begin{block}
{The main question of this talk:}
In what sense is it true that if $f$ is smoother than $g$,
then $\qd f$ is smaller than $\qd g$?
\end{block}
We shall restrict attention to functions on the circle $\Circ$
and the line $\Rl$.
\end{frame}


\begin{frame}
\frametitle{Revision of Classical Fourier Analysis and Notation}
We define $\Circ = \{\zeta \in \Cplx\;:\;|\zeta| = 1\}$.
Let $z:\Circ\to\Circ$ be the identity function. 
Denote the normalised Haar (or arc length) measure on $\Circ$ by $\ha$. 

For $f \in L^1(\Circ,\ha)$, define for $n \in \Itgr$,
\begin{equation*}
    \widehat{f}(n) := \int_\Circ z^{-n} f\;d\ha.
\end{equation*}

Recall that any $f \in L^2(\Circ,\ha)$ can be written as
\begin{equation*}
    f = \sum_{n \in \Itgr} \widehat{f}(n)z^n.
\end{equation*}
The sum converges in the $L^2$ sense. This effects
an isometric isomorphism between $L^2(\Circ)$ and $\ell^2(\Itgr)$.
\end{frame}

\begin{frame}
\frametitle{Revision of Classical Fourier Analysis and Notation}
The closed linear span of $\{z^n\}_{n=0}^\infty$ in $L^p$ is denoted
$H^2(\Circ)$, and the orthogonal complement is denoted $H^2_-(\Circ)$.

We define the space of polynomials $\mathcal{P}(\Circ)$
to be the finite linear span of $\{z^n\}_{n\in \Itgr}$. 
$P_A(\Circ) = \operatorname{span}\{z^n\}_{n\geq 0}$. 
\end{frame}



\begin{frame}
\frametitle{Differentials on $\Circ$}

 The Dirac
operator on the circle is $\D = \frac{1}{i}\frac{d}{d\theta}$, i.e.,
$\D(z^n) = nz^n$. So we may describe $F = \sgn\D$ as the \emph{Hilbert transform}, 
\begin{equation*}
    F\left(\sum_{n\in \Itgr} a_nz^n\right) = \sum_{n\in \Itgr} \sgn(n)a_nz^n.
\end{equation*}

\end{frame}




\begin{frame}
\frametitle{Differentials on $\Circ$}
Hence for $f \in L^2(\Circ)$, 
\begin{equation*}
    Ff = \varphi * f
\end{equation*} 
where
\begin{equation*}
    \varphi = \sum_{n\in\Itgr} \sgn(n)z^n = \frac{1}{1-z}-\frac{z^{-1}}{1-z^{-1}} = \frac{2}{1-z}.
\end{equation*}

   Thus, 
\begin{align*}
    (\qd f)g &= ([F,f]g)(t) \\
    &= 2\lim_{\varepsilon \to 0} \int_{|\tau-t| > \varepsilon} \frac{f(t)-f(\tau)}{t-\tau}g(\tau)\;d\ha(\tau).
\end{align*} 
\end{frame}


\begin{frame}
\frametitle{Finite rank differentials}
Let $f:\Circ\rightarrow \Cplx$. The strictest
condition we can put on the smoothness
of $f$ is that $f$ is a rational function.
The strictest condition we can put on the size of $\qd f$
is that $\qd f$ is finite rank.
These two conditions are equivalent.
\begin{theorem}[Kronecker]
    If $f:\Circ\rightarrow \Cplx$, then $\qd f$ is finite rank
    if and only if $f$ is a rational function.
\end{theorem}   
\end{frame}

\begin{frame}
\frametitle{Bounded differentials}
Let $f:\Circ\rightarrow \Cplx$. The weakest condition
that we can place on $\qd f$ is that $\qd f$ is bounded.
\begin{definition}
    Let $f:\Circ\rightarrow \Cplx$ be measureable. We say that $f$ is of \emph{bounded mean
    oscillation} if for an arc $I \subseteq \Circ$, define
    \begin{equation*}
        f_I = \frac{1}{\ha(I)}\int_I f\;d\ha
    \end{equation*}
    and
    \begin{equation*}
        \sup_{I} \frac{1}{\ha(I)}\int_I |f-f_I|\;d\ha < \infty
    \end{equation*}
    where the supremum runs over all arcs $I$.
    The set of functions with bounded mean oscillation is denoted $\BMO(\Circ)$.
\end{definition}
\end{frame} 

\begin{frame}
\frametitle{Bounded differentials}
\begin{theorem}[Nehari]
    Let $f:\Circ\rightarrow \Cplx$. Then $\qd f$ is bounded
    if and only if $f \in \BMO(\Circ)$.
\end{theorem}
\end{frame}

\begin{frame}
\frametitle{Compact differentials}
We define the space $\VMO(\Circ)$:
\begin{definition}
    We say that $f \in \VMO(\Circ)$ if $f \in \BMO(\Circ)$
    and
    \begin{equation*}
        \lim_{\ha(I)\rightarrow 0} \frac{1}{\ha(I)}\int_I |f-f_I|\;d\ha = 0.
    \end{equation*}
\end{definition}
\begin{theorem}
    If $f:\Circ\rightarrow \Cplx$, then $\qd f$ is compact if and only if $f \in \VMO(\Circ)$.
\end{theorem}
\end{frame}

\begin{frame}
\frametitle{Can we do better?}
We seek a more precise characterisation of the relationship between
the smoothness
of $f$ and the size of $\qd f$. To this end, we define the \emph{Besov classes} $B^s_{pq}$.
\end{frame}

\begin{frame}
\frametitle{Definition of $B_{pq}^s$} 
We define a sequence of polynomials $\{W_n\}_{n\in \Itgr}$ on
$\Circ$ as follows.
\begin{enumerate}
    \item{} $W_0 = z^{-1}+1+z$.
    \item{} For $n > 0$ and $k > 0$,
    \begin{equation*}
        \widehat{W}_n(k) = \begin{cases}
            1,\text{ if }k = 2^n\\
            \text{a linear function on }[2^{n-1},2^n]\text{ and }[2^n,2^{n+1}]\\
            0,\text{ otherwise.}
        \end{cases}
    \end{equation*}
    and $\widehat{W}_n(-k) = \widehat{W}_n(k)$, and $\widehat{W}_n(0) = 0$.
\end{enumerate}
\end{frame}

\begin{frame}
\frametitle{Definition of $B^s_{pq}$}
\begin{definition}
    For an integrable function $\varphi$ on $\Circ$, we say that $f \in B_{pq}^s(\Circ)$
    if
    \begin{equation*}
        \sum_{n\in \Itgr} 2^{|n|sp} \|W_n*\varphi\|_{p}^q < \infty.
    \end{equation*}
\end{definition}
\end{frame}

\begin{frame}
\frametitle{Quantised differentials of Schatten-Von Neumann class}
\begin{theorem}[Peller]
    Let $f$ be a measurable function on $\Circ$
    and $p > 0$. Then $\qd f \in \mathcal{L}^p$ if and only
    if $f \in B_{pp}^{1/p}(\Circ)$.
\end{theorem}
\end{frame}


\begin{frame}
\frametitle{Interpolation}
    Astonishingly, since we know sufficient conditions for $\qd \varphi \in \mathcal{L}^1$
    and for $\qd \varphi$ to be compact, we can derive sufficient conditions
    for $\qd \varphi \in \mathcal{L}^p$. 
    \begin{definition}
        Let $\Ban$ be the category of Banach spaces with morphisms as
        bounded linear maps.
        
        An interpolation pair is a pair of Banach spaces $(X_0,X_1)$,
        such that $X_0$ and $X_1$ are continuously embedded in a topological
        vector space $Y$. A morphism of interpolation pairs, written
        $T:(X_0,X_1)\rightarrow (Y_0,Y_1)$ is a continuous linear map
        $T:X_0+X_1\rightarrow Y_0+Y_1$, such that $T$ restricts
        to a linear map $T:X_0\rightarrow Y_0$
        and $T:X_1\rightarrow Y_1$. $\Ban_1$ is the category of interpolation
        pairs with these morphisms.
        
        An interpolation functor is a functor $F:\Ban_1\rightarrow \Ban$.
    \end{definition}
\end{frame}

\begin{frame}
\frametitle{Interpolation}
We claim that there exists an interpolation functor $F$ such that
\begin{equation*}
    F(\mathcal{L}^1,K) = \mathcal{L}^{p,q}.
\end{equation*}
Hence, if $\varphi \in F(B_{11}^1,\VMO(\Circ))$, then
\begin{equation*}
    \qd \varphi \in \mathcal{L}^{p,q}.
\end{equation*}
\end{frame}

\begin{frame}
\frametitle{Interpolation}
In fact, we have the following,
\begin{theorem}[Peller]
    Let $\varphi \in B_{pp}^{1/p}(\Circ)$. Then $\qd \varphi \in \mathcal{L}^p$.
\end{theorem}
This is true for all $p \in (1,\infty)$, and in fact the converse is true.
\end{frame}


\begin{frame}
\frametitle{Summary}

Let us see how to prove these results.
In summary we have:
\begin{tabular}{l | r}
$f$ & $\qd f$\\
\hline
Rational & Finite rank\\
$B_{pp}^{1/p}$ & $\mathcal{L}^p$\\
$K(B_{11}^1,VMO)_{(1-p)^{-1},q}$ & $\mathcal{L}^{p,q}$\\
$\VMO$ & Compact\\
$\BMO$ & Bounded
\end{tabular}
\\
You should now be convinced that the statement ``If $f$ is smoother than $g$,
then $\qd f$ is smaller than $\qd g$" has at least some rigorous justification
for functions on $\Circ$.
\end{frame}


\begin{frame}
\frametitle{Quantised differentials and classical derivatives}
Recall that $17$th century differentials were supposed to have the property
that 
\begin{equation*}
    df = f'dx
\end{equation*}
provided that sufficiently small infinitesimals are ignored. 


We can now interpret this in a rigorous sense. First, we replace
infinitesimals with quantised differentials. To ``ignore sufficiently small infinitesimals" means
that we work modulo some ideal of compact operators.
\end{frame}


\begin{frame}
\frametitle{Quantised differentials and classical derivatives}
Let $\mathcal{L}_0^{p,\infty}$ be the closure in $\mathcal{L}^{p,\infty}$
of the ideal of finite rank operators. 
\begin{theorem}[Connes]
    Let $f \in C(\Circ)$ be such that $\qd f \in \mathcal{L}^{p,\infty}$ and $\varphi \in C^\infty(\sigma(f))$, and $p \in [1,\infty)$. Then
    \begin{equation*}
        \qd \varphi(f) \equiv \varphi'(f)\qd f\;\operatorname{mod}\;\mathcal{L}^{p,\infty}_0.
    \end{equation*}
\end{theorem}
\end{frame}



\begin{frame}
\frametitle{Quantised differentials on $\Rl$}
So far we have talked about quantised differentials exclusively on $\Circ$. However,
many of the same results hold for $\Rl$, since we have the \emph{Cayley transform}.

Let $\zeta \in \Circ\setminus \{1\}$. Let 
\begin{equation*}
    \omega(\zeta) = i\frac{1+\zeta}{1-\zeta}.
\end{equation*}

Let $g \in L^0(\Rl)$. Define for $\zeta \in \Circ$, 
\begin{equation*}
    \mathcal{C}(g)(\zeta) = \frac{\sqrt{\pi}}{2i}\frac{(g\circ \omega)(\zeta)}{1-\zeta}
\end{equation*}
This allows results on $\Circ$ to be transferred to results on $\Rl$.
\end{frame}


\begin{frame}
\frametitle{Proving these statements}
To prove all the preceding results, we need to introduce the theory of Hankel operators.
\end{frame}


\begin{frame}
\frametitle{Hankel Operators}
A \emph{Hankel matrix} is an infinite matrix $\{a_{j+k}\}_{j,k \geq 0}$,
\begin{equation*}
    \begin{pmatrix}
        a_0 & a_1 & a_2 & a_3 & \cdots\\
        a_1 & a_2 & a_3 & a_4 & \cdots\\
        a_2 & a_3 & a_4 & a_5 & \cdots\\
        a_3 & a_4 & a_5 & a_6 & \cdots\\
        \vdots & \vdots & \vdots & \vdots & \ddots
    \end{pmatrix}
\end{equation*}
There is a direct link between Hankel operators and quantised differentials
on the circle.
\end{frame}


\begin{frame}
\frametitle{Hankel operators and quantised differentials} 
\begin{theorem}
    Let $\varphi \in L^1(\Circ)$, and $M_\varphi$
    is the densely defined pointwise multiplication operator
    on $L^2(\Circ)$. Let $\Proj_-$ be the projection,
    \begin{equation*}
        \Proj_-\left(\sum_{n \in \Itgr} a_nz^n\right) = \sum_{n < 0} a_nz^n.
    \end{equation*}
    Then $H_\varphi := \Proj_-M_\varphi|_{H^2(\Circ)}$ is a Hankel matrix
    when represented in the bases $\{z^n\}_{n=0}^\infty$ and $\{z^{n}\}_{n < 0}$
    with $(j,k)$th entry $\widehat{\varphi}(-j-k)$ for $j \geq 0$ and $k \geq 1$.
\end{theorem}   
\end{frame}

\begin{frame}
\frametitle{Hankel operators and quantised differentials} 
    The link between Hankel operators and quantised differentials is provided
    by the following result:
    \begin{theorem}
        Let $\varphi \in L^1(\Circ)$. Let $\varphi_- = \Proj_-\varphi$ and $\varphi_+ = (1-\Proj_-)\varphi$. Then,
        \begin{equation*}
            \qd \varphi = 2((H_{\varphi_+})^*-H_{\varphi_-})
        \end{equation*}
    \end{theorem}
\end{frame}

\begin{frame}
\frametitle{Symbol of a Hankel Operator}
\begin{definition}
    Let $\varphi \in L^1(\Circ)$. Let $\Gamma_\varphi$
    be the infinite Hankel matrix with $(j,k)$th entry $\widehat{\varphi}(j+k)$.
\end{definition}
From now on, it suffices to study the properties of $\Gamma_\varphi$. 
\end{frame}

\begin{frame}
\frametitle{Bounded Hankel Operators}
\begin{theorem}[Nehari]
    The Hankel operator $\Gamma_\varphi$ defines a bounded linear
    operator on $\ell^2(\Ntrl)$ if and only if $\varphi \in \BMO(\Circ)$.
\end{theorem}
\end{frame}

\begin{frame}
\frametitle{A result of Fefferman}
    To begin proving the Nehari theorem, we need the following result
    of C. Fefferman.
    \begin{theorem}
        The set $\BMO(\Circ)$ is exactly
        \begin{equation*}
            \BMO(\Circ) = \{f+\Proj_-g\;:\;f,g \in L^\infty(\Circ)\}.
        \end{equation*}
    \end{theorem}
\end{frame}

\begin{frame}
\frametitle{The Boundedness of Hankel operators}
\begin{theorem}[Nehari]
    Let $T = \{\alpha_{j+k}\}_{j,k\geq 0}$ be an infinite Hankel
    matrix defined by the sequence $\alpha$. $T$ defines
    a bounded linear operator on $\ell^2(\Ntrl)$ if and only if 
    there exists a function $\psi \in L^\infty(\Circ)$ such that
    $\alpha_n = \widehat{\psi}(n)$ for all $n\geq 0$,
    and
    \begin{equation*}
        \|T\| \leq \|\psi\|_\infty.
    \end{equation*}
\end{theorem}
\begin{proofs}
    Suppose that there exists such a $\psi$. Let $a$ and $b$ be finitely supported
    sequences, and compute the inner product:
    \begin{equation*}
        (Ta,b) = \sum_{j,k \geq 0} \alpha_{j+k}a_j\overline{b_k}.
    \end{equation*} 
    
\end{proofs}
\end{frame}

\begin{frame}
\begin{proofs}[Proof (Cont.)]
    Let 
    \begin{equation*}
        f = \sum_{n=0}^\infty a_n z^n,\;g = \sum_{n=0}^\infty \overline{b_n} z^n
    \end{equation*}
    and let $q = fg$. Hence,
    \begin{align*}
        (Ta,b) &= \sum_{j,k\geq 0} a_j\overline{b_k} \widehat{\psi}(j+k)\\
               &= \sum_{n\geq 0} \widehat{\psi}(n)\sum_{k=0}^n a_k\overline{b_{n-k}}\\
               &= \sum_{n\geq 0} \widehat{\psi}(n)\widehat{q}(n)\\
               &= \int_\Circ \psi(\zeta)q(\overline{\zeta})\;d\ha(\zeta).
    \end{align*} 
\end{proofs}
\end{frame}

\begin{frame}
\begin{proofs}[Proof (Cont.)]
    Hence,
    \begin{equation*}
        |(Ta,b)| \leq \|\psi\|_\infty \|q\|_1 \leq \|\psi\|_\infty \|f\|_2\|g\|_2 = \|\psi\|_\infty \|a\|_2\|b\|_2.
    \end{equation*}
    Hence $T$ is bounded on $\ell^2(\Ntrl)$.
\end{proofs}
\end{frame}

\begin{frame}
\begin{proofs}[Proof (Cont.)]
    Now we prove the converse. Suppose that $T$ is bounded on $\ell^2(\Ntrl)$. 
    
    Define $\mathcal{L}$ on $\mathcal{P}_A(\Circ)$
    \begin{equation*}
        \mathcal{L}q = \sum_{n\geq 0} \alpha_n\widehat{q}(n).
    \end{equation*}
    
    Assume $\alpha \in \ell^1(\Ntrl)$. In this case, $\mathcal{L}$
    is continuous on $H^1(\Circ)$. Let $q \in H^1(\Circ)$
    with $\|q\|_1 \leq 1$. Then $q = fg$ for $f,g\in H^2(\Circ)$
    with $\|f\|_2,\|g\|_2 \leq 1$. Hence,
    \begin{align*}
        \mathcal{L}q &= \sum_{m \geq 0} \alpha_m\widehat{q}(m)\\
        &= \sum_{m\geq 0} \alpha_m \sum_{j=0}^m \widehat{f}(j)\widehat{g}(m-j).
    \end{align*}
    
\end{proofs}    
\end{frame}

\begin{frame}
\begin{proofs}[Proof (Cont.)]
    Hence, if we define $a_j = \widehat{f}(j)$ and $b_j = \overline{\widehat{b}(j)}$,
    \begin{align*}
        \mathcal{L}q &= \sum_{m\geq 0} \alpha_m \sum_{j=0}^m \widehat{f}(j)\widehat{g}(m-j)\\
        &= (Ta,b).
    \end{align*}
    Thus,
    \begin{equation*}
        |\mathcal{L}q| \leq \|T\|\|f\|_2\|g\|_2 \leq \|T\|.
    \end{equation*}
\end{proofs} 
\end{frame}

\begin{frame}
\begin{proofs}[Proof (Cont.)]
    Now let $\alpha$ be an arbitrary sequence such that $\|T\|$ is bounded. 
    
    Let $0 < r < 1$. Define the sequence $\alpha^{(r)}$
    by
    \begin{equation*}
        \alpha^{(r)}_j = r^j\alpha_j
    \end{equation*}
    for $j \geq 0$. 
    
    Let $T^{(r)}$ be the Hankel matrix $\{\alpha^{(r)}_{j+k}\}_{j,k \geq 0}$. 
    
    Let $D_r$ be the operator of multiplication by $\{r^j\}_{j\geq 0}$ on
    $\ell^2(\Ntrl)$. Then we can compute that $T^{(r)} = D_rTD_r$. 
    Hence $T^{(r)}$ is bounded, and $\|T^{(r)}\| \leq \|T\|$. 
\end{proofs}
\end{frame}

\begin{frame}
\begin{proofs}[Proof (Cont.)]
    Let
    \begin{equation*}
        \mathcal{L}_rq = \sum_{n\geq 0} \alpha^{(r)}_n\widehat{q}(n).
    \end{equation*}
    Since $\alpha^{(r)} \in \ell^2(\Ntrl)$, we have already
    proved that $\|\mathcal{L}\|_{H^1\to\Cplx} \leq \|T^{(r)}\| \leq \|T\|$.
    
    Now the functionals $\{\mathcal{L}^{(r)}\}_{r \in (0,1)}$ are
    uniformly bounded and converge strongly to $\mathcal{L}$. 
    
    Hence, $\mathcal{L}$ is continuous.
\end{proofs}
\end{frame}

\begin{frame}
\begin{proof}[Proof (Cont.)]
    We have shown that $\mathcal{L}$ is bounded on $H^1(\Circ)$
    with norm bounded by $\|T\|$. Hence
    by the Hahn Banach theorem
    there is $\psi \in L^\infty$ such that $\mathcal{L}q = (\psi,q)$,
    and $\|\psi\|_\infty \leq \|T\|$.
    
    This completes the proof.
\end{proof}
\end{frame}

\begin{frame}
\frametitle{Reformulation in terms of $H_\varphi$}
Recall that we defined
\begin{equation*}
    H_\varphi = \Proj_- M_\varphi|_{H^2(\Circ)}.
\end{equation*}
For $\varphi \in L^1(\Circ)$. This has matrix
representation $\{\widehat{\varphi}(-j-k)\}_{j\geq 1,k\geq 0}$. 
Hence, $H_\varphi$ is bounded if and only if there exists $\psi \in L^\infty(\Circ)$
such that
\begin{equation*}
    \widehat{\varphi}(-j) = \widehat{\psi}(j).
\end{equation*}
And hence $\Proj_- \varphi \in \BMO(\Circ)$. Conversely,
if $\Proj_-\varphi \in \BMO(\Circ)$, then such a $\psi$
exists. 
\end{frame}

\begin{frame}
\frametitle{Back to quantised derivatives}
Recall that
\begin{equation*}
    \qd \varphi = 2((H_{\varphi_-})^*-H_{\varphi_+}).
\end{equation*}
Since the two Hankel operators act on orthogonal complements, we
see that $\qd \varphi$ is bounded if and only if $H_{\varphi_+}$
and $H_{\varphi_-}$ are bounded. Hence
\begin{theorem}
    $\qd \varphi$ is bounded if and only is $\varphi \in \BMO(\Circ)$.
\end{theorem}
\end{frame}

\begin{frame}
\frametitle{Finite rank Hankel operators}
Now let us prove Kronecker's theorem.
\begin{theorem}
    Let $\varphi \in \BMO(\Circ)$.
    $\Gamma_\varphi$ is of finite rank if and only if $\varphi$ is a rational
    function.
\end{theorem}
Recall that $\Gamma_\varphi$ is the Hankel matrix with $(j,k)$th entry $\widehat{\varphi}(j+k)$.
\begin{proofs}
    First assume that $\Gamma_\varphi$ is a finite rank operator on $\ell^2(\Ntrl)$. 
    Let $\rk(\Gamma_\varphi) = n$. Then the first $n+1$
    columns of $\Gamma_\varphi$ are linearly dependent. 
\end{proofs}
\end{frame} 
 
\begin{frame}
\begin{proofs}[Proof (Cont.)]
    Let $F$ be the operator on $H^2(\Circ)$ that is ``forward shift", that is
    \begin{equation*}
        F\left(\sum_{n=0}^\infty a_nz^n\right) = \sum_{n=0}^\infty a_{n+1} z^n.
    \end{equation*}
    and $B$ is ``backward shift", so $Bf(\zeta) = \zeta f(\zeta)$. 
    
    The first $n+1$ columns of $\Gamma_\varphi$ being linearly dependent
    means that there are non-trivial scalars $\{c_j\}_{j=0}^n$
    such that 
    \begin{equation*}
        c_0\varphi + c_1 F\varphi + \cdots + c_nF\varphi = 0.
    \end{equation*}
\end{proofs}
\end{frame}  

\begin{frame}
\begin{proofs}[Proof (Cont.)]
    For a function $f \in L^2(\Circ)$, it is easy to see that
    \begin{equation*}
        B^n F^k f = B^{n-k}f - B^{n-k}\sum_{j=0}^{k-1} \widehat{f}(j)z^j.
    \end{equation*}
    Hence, since
    \begin{equation*}
        \sum_{k=0}^n c_k F^k \varphi = 0
    \end{equation*}
    we conclude that
    \begin{equation*}
        0 = \sum_{k=0}^n c_k B^n F^k \varphi = \sum_{k=0}^n c_kB^{n-k}\varphi - p,
    \end{equation*}
    where $p$ is a polynomial.
    Let $q = \sum_{k=0}^n c_{n-k}z^k$. Hence $q\varphi = p$,
    so $\varphi$ is a rational function.
\end{proofs}
\end{frame}

\begin{frame}
\begin{proofs}[Proof (Cont.)]
    Now we prove the converse. Suppose that $\varphi = p/q$
    for polynomials $p$ and $q$ such that $\deg(p) \leq n-1$
    and $\deg(q) \leq n$. Again, let
    \begin{equation*}
        q = \sum_{j=0}^n c_{n-j}z^j.
    \end{equation*}
    Hence since $\varphi q = p$, we have
    \begin{equation*}
        \sum_{j=0}^n c_j B^{n-j} \varphi = 0.
    \end{equation*}
\end{proofs}
\end{frame}

\begin{frame}
\begin{proofs}[Proof (Cont.)]
    Hence,
    \begin{align*}
        F^n \sum_{j=0}^n c_j B^{n-j} \varphi &= \sum_{j=0}^n c_jF^j\alpha \\
        &= 0.
    \end{align*}
    Which means that the first $n+1$ rows of $\Gamma_\varphi$
    are linearly dependent. 
    
    Let $m \leq n$ be the largest number such that $c_m \neq 0$. 
    Then $F^m \varphi$ is a linear combination
    of the $F^j\varphi$ with $j \leq m-1$:
    \begin{equation*}
        F^m \varphi = \sum_{j=0}^{m-1}d_j F^j \varphi,
    \end{equation*} 
    for some nontrivial choice of scalars $\{d_j\}_{j=0}^{m-1}$. 
\end{proofs}
\end{frame}

\begin{frame}
\begin{proof}[Proof (Cont.)]
    Let us proceed by induction to show that any row of $\Gamma_\varphi$
    is a linear combination of the first $m$ rows. Let $k > m$. We have
    \begin{align*}
        F^k\varphi &= F^{k-m}F^m\varphi\\
                   &= \sum_{j=0}^{m-1} d_j F^{k-m+j}\varphi
    \end{align*}
    Since for each $0 \leq j \leq m-1$, we have $k-m+j < k$,
    by the inductive hypothesis the right hand side is a linear
    combination of the first $m$ rows. Thus $\rk(\Gamma_\varphi) \leq m$.
\end{proof}
\end{frame}

\begin{frame}
\frametitle{Finite rank quantised derivatives}
It is now straightforward to show that:
\begin{theorem}
    $\qd \varphi$ is finite rank if and only if $\varphi$
    is a rational function.
\end{theorem}
\end{frame}

\begin{frame}
\frametitle{Compactness of quantised derivatives}
Recall that in classical analysis we could claim that
$f$ is continuous if and only if $df$ is infinitesimal. 

Since
\begin{equation*}
    \|\qd \varphi\| = \|[\sgn(\D),M_\varphi]\| \leq 2\|\varphi\|_\infty,
\end{equation*}
and $\varphi$ a rational function implies that $\qd \varphi$
is of finite rank, we obtain
\begin{theorem}
    Let $\varphi \in C(\Circ)$. Then $\qd \varphi$ is compact (i.e.,
    infinitesimal).
\end{theorem}
\end{frame}

\begin{frame}
\frametitle{Compact Hankel operators}
It can be proved that
\begin{theorem}
    Let $\varphi \in L^2(\Circ)$. Then $H_\varphi$ is compact
    if and only if $\Proj_-\varphi \in \VMO(\Circ)$.
\end{theorem}
Consequently $\qd \varphi$ is compact if and only if $\varphi \in \VMO(\Circ)$.
\end{frame}

\begin{frame}
\frametitle{Trace class Hankel operators}
\begin{theorem}
    Let $\varphi$ be a function holomorphic
    in the unit disc. Then
    $\Gamma_\varphi \in \mathcal{L}^1$ if and only if $\varphi \in B_{11}^1(\Circ)$.
\end{theorem}
Recall that $\varphi \in B_{11}^1(\Circ)$ means
that
\begin{equation*}
    \sum_{n\geq 0} 2^{|n|} \|W_n*\varphi\|_1 < \infty.
\end{equation*}
\end{frame}

\begin{frame}
\frametitle{A lemma about $\mathcal{L}^1$ norms}
\begin{lemma}
    Let $f \in \mathcal{P}_A(\Circ)$, an analytic polynomial
    of degree at most $m$. Then
    \begin{equation*}
        \|\Gamma_f\|_1 \leq (m+1)\|f\|_1.
    \end{equation*}
\end{lemma}
\begin{proofs}
    Let $\zeta \in \Circ$.
    Define the two vectors $x_\zeta,y_\zeta \in \ell^2(\Ntrl)$, 
    \begin{align*}
        x_\zeta &= (1,\zeta,\zeta^2,\ldots,\zeta^m,0,0,\ldots)\\
        y_\zeta &= f(\zeta)(1,\zeta^{-1},\zeta^{-2},\ldots,\zeta^{-m},0,0,\ldots).
    \end{align*}
    Furthermore define the rank one operator $A_\zeta = y_\zeta(x,x_\zeta)$. 
\end{proofs}
\end{frame}

\begin{frame}
\begin{proofs}[Proof (Cont.)]
    We can see that the $(j,k)$th entry of the matrix for $A_\zeta$ is
    \begin{equation*}
        f(\zeta)\zeta^{-j-k}.
    \end{equation*}
    Hence we have an entrywise equality,
    \begin{equation*}
        \Gamma_f = \int_\Circ A_\zeta\;d\ha(\zeta).
    \end{equation*}
    But since these matrices only have finitely many non-zero entries, 
    this can be considered as a Bochner integral. Therefore,
    \begin{equation*}
        \|\Gamma_f\| \leq \int_\Circ \|A_\zeta\|_1\;d\ha(\zeta).
    \end{equation*}
\end{proofs}
\end{frame}

\begin{frame}
\begin{proof}[Proof (Cont.)]
Since $A_\zeta$ is rank one, we can compute its $\mathcal{L}^1$ norm,
\begin{equation*}
    \|A_\zeta\|_1 = \|x_\zeta\|_2\|y_\zeta\|_2 = (m+1)|f(\zeta)|.
\end{equation*}
Therefore,
\begin{equation*}
    \|\Gamma_f\|_1 \leq \int_\Circ (m+1)|f(\zeta)|\;d\ha(\zeta) = (m+1)\|f\|_1.
\end{equation*}
\end{proof}
\end{frame}

\begin{frame}
\frametitle{Trace class Hankel operators}
Finally we can prove the following:
\begin{theorem}
    Let $\varphi \in B_{11}^1(\Circ)$ and analytic in the unit disc. Then $\Gamma_\varphi \in \mathcal{L}^1$. 
\end{theorem}
\begin{proofs}
    By the definition of the sequence $\{W_n\}_{n\geq 0}$, we have
    \begin{equation*}
        \varphi = \sum_{n\geq 0} W_n*\varphi.
    \end{equation*}
    This series converges uniformly since $\varphi \in \mathcal{B}_{11}^1(\Circ) \subseteq C^1(\Circ)$.
    Hence,
    \begin{equation*}
        \Gamma_\varphi = \sum_{n\geq 0} \Gamma_{W_n*\varphi}
    \end{equation*}
    is a series converging in the operator norm topology.
\end{proofs} 
\end{frame}

\begin{frame}
\begin{proof}[Proof (Cont.)]
    Hence,
    \begin{equation*}
        \|\Gamma_\varphi\|_1 \leq \sum_{n\geq 0} \|\Gamma_{W_n * \varphi}\|_1.
    \end{equation*}
    But since $W_n*\varphi$ is a polynomial of degree at most $2^{n+1}$, we get
    \begin{equation*}
        \|\Gamma_\varphi\|_1 \leq \sum_{n\geq 0} 2^{n+1} \|W_n*\varphi\|_1.
    \end{equation*}
\end{proof}
\end{frame}

\begin{frame}
\frametitle{Trace class quantised differentials}
As a consequence, we obtain:
\begin{theorem}
    Let $\varphi \in B_{11}^1(\Circ)$, then $\qd \varphi \in \mathcal{L}^1$.
\end{theorem}
The converse is also true. In fact, it is true that
\begin{equation*}
    \frac{1}{6}\sum_{n\geq 0} 2^n\|W_n*\varphi\|_1 \leq \|\Gamma_\varphi\|_1 \leq 2\sum_{n\geq 0}2^n \|W_n*\varphi\|_1.
\end{equation*}
\end{frame}

\begin{frame}
\frametitle{The End}
In summary:
\begin{enumerate}
    \item{} Quantum mechanics causes us to look for generalisations of differential
    geometry in non-commutative algebra.
    \item{} If we define non-commutative spaces as spectral triples, a new object appears: the quantised differential.
    \item{} The quantised differential has meaning in analysis on $\Circ$
    and $\Rl$. 
    \item{} The quantised differential has many properties that we would 
    anticipate for ``infinitesimal differences".
\end{enumerate}
Thank you for listening!
\end{frame}
\end{document}
